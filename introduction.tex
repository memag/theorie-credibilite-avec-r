%%% Copyright (C) 2018 Vincent Goulet
%%%
%%% Ce fichier fait partie du projet
%%% «Théorie de la crédibilité avec R»
%%% http://github.com/vigou3/theorie-credibilite-avec-r
%%%
%%% Cette création est mise à disposition selon le contrat
%%% Attribution-Partage dans les mêmes conditions 4.0
%%% International de Creative Commons.
%%% http://creativecommons.org/licenses/by-sa/4.0/

\chapter*{Introduction}
\addcontentsline{toc}{chapter}{Introduction}
\markboth{Introduction}{Introduction}

Ce document est le fruit de la mise en commun d'exercices colligés au
fil du temps pour nos cours de théorie de la crédibilité à
l'Université Laval et à l'Université Concordia. Nous ne sommes
toutefois pas les uniques auteurs des exercices; certains ont, en
effet, été rédigés par les Docteurs François Dufresne et Jacques
Rioux, entre autres. Quelques exercices proviennent également
d'anciens examens de la Society of Actuaries et de la Casualty
Actuarial Society.

C'est d'ailleurs afin de ne pas usurper de droits d'auteur que ce
document est publié selon les termes du contrat Paternité-Partage des
conditions initiales à l'identique 2.5 Canada de Creative Commons. Il
s'agit donc d'un document «libre» que quiconque peut réutiliser et
modifier à sa guise, à condition que le nouveau document soit publié
avec le même contrat.

Le premier chapitre, tiré de \cite{Goulet:masters}, trace l'historique et
l'évolution de la théorie de la crédibilité, de ses origines jusqu'au
début des années 1990. Les chapitres \ref{chap:stabilite} à
\ref{chap:bstraub} --- qui correspondent aux chapitres de notre cours
--- sont quant à eux uniquement constitués d'exercices. Dans cette
seconde édition, les réponses des exercices se trouvent à la fin des
chapitres, alors que les solutions complètes se trouvent à l'annexe
\ref{chap:solutions}. De plus, à la fin de chaque chapitre
d'exercices, on trouvera une liste d'exercices suggérés dans
\cite{Klugman:lossmodels:2e:2004,Klugman:lossmodels:3e:2008}.

Un tableau synoptique des principaux résultats de crédibilité exacte
se trouve à l'annexe \ref{chap:formules}. L'annexe
\ref{chap:distributions}, présente la paramétrisation des lois de
probabilité utilisée dans les exercices. En cas de doute, le lecteur
est invité à la consulter. Il y trouvera également l'espérance, la
variance et la fonction génératrice des moments (lorsqu'elle existe)
des lois de probabilités rencontrées dans ce document. Enfin, l'annexe
\ref{chap:normale} contient un tableau des quantiles de la loi
normale

Nous tenons à remercier M.~Mathieu Pigeon pour sa précieuse
collaboration lors de la préparation de ce document, ainsi que tous
les auxiliaires d'enseignement ayant, au fil des années, contribué à
la rédaction d'exercices et de solutions.

\begin{flushright}
  Hélène Cossette \url{<helene.cossette@act.ulaval.ca>} \\
  Vincent Goulet \url{<vincent.goulet@act.ulaval.ca>} \\
  Québec, décembre 2007
\end{flushright}

%%% Local Variables:
%%% mode: latex
%%% TeX-engine: xetex
%%% TeX-master: "theorie-credibilite-avec-r"
%%% coding: utf-8
%%% End:
