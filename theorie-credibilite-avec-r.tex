%%% Copyright (C) 2018 Vincent Goulet
%%%
%%% Ce fichier fait partie du projet
%%% «Théorie de la crédibilité avec R»
%%% http://github.com/vigou3/theorie-credibilite-avec-r
%%%
%%% Cette création est mise à disposition selon le contrat
%%% Attribution-Partage dans les mêmes conditions 4.0
%%% International de Creative Commons.
%%% http://creativecommons.org/licenses/by-sa/4.0/

\documentclass[letterpaper,11pt,x11names,english,french]{memoir}
  \usepackage{natbib,url}
  \usepackage{babel}
  \usepackage[autolanguage]{numprint}
  \usepackage{amsmath,amsthm}
  \usepackage[noae]{Sweave}
  \usepackage{graphicx}
  \usepackage{rotating}                % env. sideways pour annexe A
  \usepackage{framed}                  % env. snugshade*, oframed
  \usepackage[absolute]{textpos}       % éléments des pages de titre
  \usepackage[shortlabels]{enumitem}   % configuration listes
  \usepackage{relsize}                 % \smaller et al.
  \usepackage{manfnt}                  % \mantriangleright (puce)
  \usepackage{metalogo}                % \XeLaTeX logo
  \usepackage{fontawesome}             % icônes \fa*
  \usepackage{awesomebox}              % boites info, important, etc.
  \usepackage{answers}                 % exercices et solutions
  \usepackage{listings}                % code informatique
  \usepackage{pgf}                     % transparence pour couverture avant

  %%% =============================
  %%%  Informations de publication
  %%% =============================
  \title{Théorie de la crédibilité avec R}
  \author{Vincent Goulet}
  \renewcommand{\year}{2018}
  \renewcommand{\month}{01-1}
  \newcommand{\ghurl}{https://github.com/vigou3/theorie-credibilite-avec-r/}

  %%% ===================
  %%%  Style du document
  %%% ===================

  %% Polices de caractères
  \usepackage{fontspec}
  \usepackage{unicode-math}
  \defaultfontfeatures{Scale=0.92}
  \setmainfont[Ligatures=TeX,Numbers=OldStyle]{Lucida Bright OT}
  \setmathfont{Lucida Bright Math OT}
  \setmonofont{Lucida Grande Mono DK}
  \setsansfont[Scale=1.0,Numbers=OldStyle]{Myriad Pro}
  \newfontfamily\fullcaps[Letters=Uppercase,Numbers=Uppercase]{Myriad Pro}
  \usepackage[babel=true]{microtype}
  \usepackage{icomma}

  %% Couleurs
  \usepackage{xcolor}
  \definecolor{comments}{rgb}{0.7,0,0}      % commentaires
  \definecolor{link}{rgb}{0,0.4,0.6}        % liens internes
  \definecolor{url}{rgb}{0.6,0,0}           % liens externes
  \definecolor{citation}{rgb}{0,0.5,0}      % citations
  \definecolor{codebg}{named}{LightYellow1} % fond code R
  \definecolor{darkblue}{rgb}{0,0,0.7}      % pour théorème 4.1
  \colorlet{darkred}{comments}              % idem
  \colorlet{darkgreen}{citation}            % idem
  \definecolor{rouge}{rgb}{0.85,0,0.07} % rouge bandeau identitaire
  \definecolor{or}{rgb}{1,0.8,0}        % or bandeau identitaire

  %% Hyperliens
  \usepackage{hyperref}
  \hypersetup{%
    pdfauthor = \theauthor,
    pdftitle = \thetitle,
    colorlinks = true,
    linktocpage = true,
    urlcolor = {url},
    linkcolor = {link},
    citecolor = {citation},
    pdfpagemode = {UseOutlines},
    pdfstartview = {Fit}}
  \setlength{\XeTeXLinkMargin}{1pt}

  %% Affichage de la table des matières du PDF
  \usepackage{bookmark}
  \bookmarksetup{%
    open = true,
    depth = 3,
    numbered = true}

  %% Étiquettes de \autoref (redéfinitions compatibles avec babel).
  %% Attention! Les % à la fin des lignes sont importants sinon des
  %% blancs apparaissent dès que la commande \selectlanguage est
  %% utilisée... comme dans la bibliographie, par exemple.
  \addto\extrasfrench{%
    \def\appendixautorefname{annexe}%
    \def\figureautorefname{figure}%
    \def\exempleautorefname{exemple}%
    \def\exerciceautorefname{exercice}%
    \def\subfigureautorefname{figure}%
    \def\subsectionautorefname{section}%
    \def\subtableautorefname{tableau}%
    \def\tableautorefname{tableau}%
    \def\thmautorefname{théorème}%
  }

  %% Table des matières (inspirée de classicthesis.sty)
  \renewcommand{\cftchapterleader}{\hspace{1.5em}}
  \renewcommand{\cftchapterafterpnum}{\cftparfillskip}
  \renewcommand{\cftsectionleader}{\hspace{1.5em}}
  \renewcommand{\cftsectionafterpnum}{\cftparfillskip}

  %% Titres des chapitres
  \chapterstyle{hangnum}
  \renewcommand{\chaptitlefont}{\normalfont\Huge\sffamily\bfseries\raggedright}

  %% Marges, entêtes et pieds de page
  \setlength{\marginparsep}{7mm}
  \setlength{\marginparwidth}{13mm}
  \setlength{\headwidth}{\textwidth}
  \addtolength{\headwidth}{\marginparsep}
  \addtolength{\headwidth}{\marginparwidth}

  %% Titres des sections et sous-sections
  \setsecheadstyle{\normalfont\Large\sffamily\bfseries\raggedright}
  \setsubsecheadstyle{\normalfont\large\sffamily\bfseries\raggedright}
  \maxsecnumdepth{subsection}
  \setsecnumdepth{subsection}

  %% Listes. Paramétrage avec enumitem.
  \setlist[enumerate]{leftmargin=*,align=left}
  \setlist[enumerate,2]{label=\alph*),labelsep=*,leftmargin=1.5em}
  \setlist[enumerate,3]{label=\roman*),labelsep=*,leftmargin=1.5em,align=right}
  \setlist[itemize]{leftmargin=*,align=left}

  %% Options de babel
  \frenchbsetup{CompactItemize=false,%
    ThinSpaceInFrenchNumbers=true,
    ItemLabeli=\mantriangleright,
    ItemLabelii=\textendash,
    og=«, fg=»}
  \addto\captionsfrench{\def\figurename{{\scshape Fig.}}}
  \addto\captionsfrench{\def\tablename{{\scshape Tab.}}}

  %%% =========================
  %%%  Sections de code source
  %%% =========================
  \lstloadlanguages{R}
  \lstset{language=R,
    extendedchars=true,
    basicstyle=\small\ttfamily\NoAutoSpacing,
    commentstyle=\color{comments}\slshape,
    keywordstyle=\mdseries,
    showstringspaces=false}

  %%% =========================
  %%%  Nouveaux environnements
  %%% =========================

  %% Environnements d'exemples et al.
  \theoremstyle{plain}
  \newtheorem{thm}{Théorème}[chapter]
  \newtheorem{corrolaire}{Corollaire}[chapter]

  \theoremstyle{definition}
  \newtheorem*{definition}{Définition}
  \newtheorem{exemple}{Exemple}[chapter]

  \theoremstyle{remark}
  \newtheorem*{rem}{Remarque}
  \newtheorem*{rems}{Remarques}

  %% Redéfinition de l'environnement titled-frame de framed.sty avec
  %% deux modifications: épaisseur des filets réduite de 2pt à 1pt;
  %% "(suite)" plutôt que "(cont)" dans la barre de titre
  %% lorsque l'encadré se poursuit après un saut de page.
  \renewenvironment{titled-frame}[1]{%
    \def\FrameCommand{\fboxsep8pt\fboxrule1pt
      \TitleBarFrame{\textbf{#1}}}%
    \def\FirstFrameCommand{\fboxsep8pt\fboxrule1pt
      \TitleBarFrame[$\blacktriangleright$]{\textbf{#1}}}%
    \def\MidFrameCommand{\fboxsep8pt\fboxrule1pt
      \TitleBarFrame[$\blacktriangleright$]{\textbf{#1\ (suite)}}}%
    \def\LastFrameCommand{\fboxsep8pt\fboxrule1pt
      \TitleBarFrame{\textbf{#1\ (suite)}}}%
    \MakeFramed{\advance\hsize-16pt \FrameRestore}}%
  {\endMakeFramed}

  %% Encadré générique avec titre basé sur titled-frame, ci-dessus.
  %% Sert pour les listes d'objectifs et les encadrés reliés aux
  %% problèmes (mises en situation) dans les chapitres. Arguments:
  %% couleur du cadre (optionnel; noir par défaut) et titre de la
  %% boite (obligatoire).
  \newenvironment{emphbox}[2][black]{%
    \colorlet{TFFrameColor}{#1}%
    \colorlet{TFTitleColor}{white}%
    \begin{titled-frame}{\sffamily #2}%
      \setlength{\parindent}{0pt}}%
    {\end{titled-frame}}

  %% Liste d'objectifs au début des chapitres
  \newenvironment{objectifs}{%
    \begin{emphbox}{\rule[-7pt]{0pt}{20pt} Objectifs du chapitre}
      \begin{itemize}[nosep]
        \small\sffamily}%
      {\end{itemize}\end{emphbox}}

  %% Environnements de Sweave. Les environnements Sinput et Soutput
  %% utilisent Verbatim (de fancyvrb). On les réinitialise pour
  %% enlever la configuration par défaut de Sweave, puis on réduit
  %% l'écart entre les blocs Sinput et Soutput.
  \DefineVerbatimEnvironment{Sinput}{Verbatim}{}
  \DefineVerbatimEnvironment{Soutput}{Verbatim}{}
  \fvset{listparameters={\setlength{\topsep}{0pt}}}

  %% L'environnement Schunk est complètement redéfini en un hybride
  %% des environnements snugshade* et leftbar de framed.sty.
  \makeatletter
  \renewenvironment{Schunk}{%
    \setlength{\topsep}{1pt}
    \def\FrameCommand##1{\hskip\@totalleftmargin
       \vrule width 2pt\colorbox{codebg}{\hspace{3pt}##1}%
      % There is no \@totalrightmargin, so:
      \hskip-\linewidth \hskip-\@totalleftmargin \hskip\columnwidth}%
    \MakeFramed {\advance\hsize-\width
      \@totalleftmargin\z@ \linewidth\hsize
      \advance\labelsep\fboxsep
      \@setminipage}%
  }{\par\unskip\@minipagefalse\endMakeFramed}
  \makeatother

  %% Exercices et réponses
  \Newassociation{sol}{solution}{solutions}
  \Newassociation{rep}{reponse}{reponses}
  \newcounter{exercice}[chapter]
  \renewcommand{\theexercice}{\thechapter.\arabic{exercice}}
  \newenvironment{exercice}[1][]{%
    \begin{list}{}{%
        \refstepcounter{exercice}
        \ifthenelse{\equal{#1}{nosol}}{%
          \renewcommand{\makelabel}{\bfseries\theexercice}}{%
          \hypertarget{ex:\theexercice}{}
          \Writetofile{solutions}{\protect\hypertarget{sol:\theexercice}{}}
          \renewcommand{\makelabel}{%
            \bfseries\protect\hyperlink{sol:\theexercice}{\theexercice}}}
        \settowidth{\labelwidth}{\bfseries\theexercice}
        \setlength{\leftmargin}{\labelwidth}
        \addtolength{\leftmargin}{\labelsep}
        \setlist[enumerate,1]{label=\alph*),labelsep=*,leftmargin=1.5em}
        \setlist[enumerate,2]{label=\roman*),labelsep=0.5em,align=right}}
      \item}%
      {\end{list}}
  \renewenvironment{solution}[1]{%
    \begin{list}{}{%
        \renewcommand{\makelabel}{%
          \bfseries\protect\hyperlink{ex:#1}{#1}}
        \settowidth{\labelwidth}{\bfseries #1}
        \setlength{\leftmargin}{\labelwidth}
        \addtolength{\leftmargin}{\labelsep}
        \setlist[enumerate,1]{label=\alph*),labelsep=*,leftmargin=1.5em}
        \setlist[enumerate,2]{label=\roman*),labelsep=0.5em,align=right}}
      \item}
    {\end{list}}
  \renewenvironment{reponse}[1]{%
    \begin{enumerate}[label=\textbf{#1}]
    \item}%
    {\end{enumerate}}

  %%% =====================
  %%%  Nouvelles commandes
  %%% =====================

  %% Noms de fonctions, code, etc.
  \newcommand{\code}[1]{\texttt{#1}}
  \newcommand{\pkg}[1]{\textbf{#1}}

  %% Hyperlien avec symbole de lien externe juste après; second
  %% argument peut être vide pour afficher l'url comme lien
  %% [https://tex.stackexchange.com/q/53068/24355 pour procédure de
  %% test du second paramètre vide]
  \newcommand{\link}[2]{%
    \def\param{#2}%
    \ifx\param\empty
      \href{#1}{\nolinkurl{#1}~\raisebox{-0.1ex}{\smaller\faExternalLink}}%
    \else
      \href{#1}{#2~\raisebox{-0.1ex}{\smaller\faExternalLink}}%
    \fi
  }

  %% Boite additionnelle (basée sur awesomebox.sty) pour changements
  %% au fil de la lecture.
  \newcommand{\gotorbox}[1]{%
    \awesomebox{\faMapSigns}{\aweboxrulewidth}{black}{\sffamily #1}}

  %% Boite pour le nom du fichier de script correspondant au début des
  %% sections d'exemples.
  \newcommand{\scriptfile}[1]{%
    \begingroup
    \noindent
    \mbox{%
      \makebox[3mm][l]{\raisebox{-0.5pt}{\small\faChevronCircleDown}}\;%
      \smaller[1] {\sffamily Fichier d'accompagnement} {\ttfamily #1}}
    \endgroup}

  %% Lien vers GitHub dans la page de notices
  \newcommand{\viewsource}[1]{%
    \href{#1}{%
      Voir sur GitHub \raisebox{-1pt}{\footnotesize\faGithub}}}

  %% Raccourcis usuels vg
  \newcommand{\pt}{{\scriptscriptstyle \Sigma}}
  \newcommand{\Esp}[1]{E\! \left[ #1 \right]}
  \newcommand{\esp}[1]{E [ #1 ]}
  \newcommand{\Var}[1]{\operatorname{Var}\! \left[ #1 \right]}
  \newcommand{\var}[1]{\operatorname{Var} [ #1 ]}
  \newcommand{\Cov}{\operatorname{Cov}}
  \ifxetex
    \newcommand{\mat}[1]{\symbf{#1}}
  \else
    \newcommand{\mat}[1]{\mathbf{#1}}
  \fi
  \ifxetex
    \newcommand{\Z}{\symbb{Z}}
  \else
    \newcommand{\Z}{\mathbb{Z}}
  \fi

  %%% =======
  %%%  Varia
  %%% =======

  %%% Sous-figures
  \newsubfloat{figure}

  %%% Style de la bibliographie
  \bibliographystyle{francais}

  %% Longueurs pour la composition des pages couvertures avant et
  %% arrière.
  \newlength{\banderougewidth} \newlength{\banderougeheight}
  \newlength{\bandeorwidth}    \newlength{\bandeorheight}
  \newlength{\imageheight}
  \newlength{\logoheight}
  \newlength{\gapwidth}

  %% Longueurs et type de colonne pour les tableaux du chapitre 1.
  \newlength{\tablewidth}
  \setlength{\tablewidth}{\textwidth}
  \addtolength{\tablewidth}{-2\parindent}
  \newcolumntype{Y}{>{\centering\arraybackslash$}X<{$}}

  %% Aide pour la césure
  \hyphenation{%
  }

%  \includeonly{}

\begin{document}

\frontmatter

\pagestyle{empty}

\input{couverture-avant}
\null\cleardoublepage           % cf. section 2.2 textpos.pdf

%%% Copyright (C) 2018 Vincent Goulet
%%%
%%% Ce fichier fait partie du projet
%%% «Théorie de la crédibilité avec R»
%%% http://github.com/vigou3/theorie-credibilite-avec-r
%%%
%%% Cette création est mise à disposition selon le contrat
%%% Attribution-Partage dans les mêmes conditions 4.0
%%% International de Creative Commons.
%%% http://creativecommons.org/licenses/by-sa/4.0/

\begingroup
\calccentering{\unitlength}
\begin{adjustwidth*}{\unitlength}{-\unitlength}
  \setlength{\parindent}{0pt}
  \setlength{\parskip}{\baselineskip}
  \small

  \raisebox{-2.5mm}{%
    \includegraphics[height=7mm,keepaspectratio=true]{by-sa}} %
  {\theauthor}, {\year}

  {\textcopyright} {\year} par {\theauthor}. «\thetitle»
  est mis à disposition selon le contrat
  \href{http://creativecommons.org/licenses/by-sa/4.0/deed.fr}{%
    Attribution-Partage dans les mêmes conditions 4.0 International} de
  Creative Commons. En vertu de ce contrat, vous êtes libre de:
  \begin{itemize}
  \item \textbf{partager} --- reproduire, distribuer et communiquer
    l'œuvre;
  \item \textbf{remixer} --- adapter l'œuvre;
  \item utiliser cette œuvre à des fins commerciales.
  \end{itemize}
  Selon les conditions suivantes:

  \begin{tabularx}{\linewidth}{@{}lX@{}}
    \raisebox{-9mm}[0mm][13mm]{%
      \includegraphics[height=11mm,keepaspectratio=true]{by}} &
    \textbf{Attribution} --- Vous devez créditer l'œuvre, intégrer
    un lien vers le contrat et indiquer si des modifications ont été
    effectuées à l'œuvre. Vous devez indiquer ces informations par
    tous les moyens possibles, mais vous ne pouvez suggérer que
    l'Offrant vous soutient ou soutient la façon dont vous avez utilisé
    son œuvre. \\
    \raisebox{-9mm}{\includegraphics[height=11mm,keepaspectratio=true]{sa}}
    & \textbf{Partage dans les mêmes conditions} --- Dans le cas où vous
    modifiez, transformez ou créez à partir du matériel composant
    l'œuvre originale, vous devez diffuser l'œuvre modifiée dans
    les même conditions, c'est à dire avec le même contrat avec lequel
    l'œuvre originale a été diffusée.
  \end{tabularx}

  Les exercices sont dérivés de \citet{Cossette:credibilite:2008}, un
  document sous contrat
  \href{http://creativecommons.org/licenses/by-sa/2.5/ca/}{%
    CC BY-SA 2.5 Canada}.

  \textbf{Code source} \\
  \viewsource{\ghurl}

  \textbf{Couverture} \\
  Bouquetin des Alpes (\emph{Capra ibex}) du parc national autrichien
  du Hohe Tauern avec, en arrière-plan, le glacier Pasterze, le plus
  grand d'Autriche. À la tombée du jour, lorsque les randonneurs ont
  quitté la région, ces farouches chèvres de montagne s'approchent
  parfois plus qu'à l'habitude des sentiers.

  Crédit photo: {\textcopyright} Bernd Thaller,
  \href{https://creativecommons.org/licenses/by-sa/3.0/at/deed.en}{CC
    BY-SA 3.0 Autriche}, via
  \href{https://commons.wikimedia.org/w/index.php?curid=40071439}{Wikimedia
    Commons}.
\end{adjustwidth*}
\endgroup

%%% Local Variables:
%%% mode: latex
%%% TeX-engine: xetex
%%% TeX-master: "theorie-credibilite-avec-r"
%%% coding: utf-8
%%% End:

\clearpage

\pagestyle{companion}

%%% Copyright (C) 2018 Vincent Goulet
%%%
%%% Ce fichier fait partie du projet
%%% «Théorie de la crédibilité avec R»
%%% http://github.com/vigou3/theorie-credibilite-avec-r
%%%
%%% Cette création est mise à disposition selon le contrat
%%% Attribution-Partage dans les mêmes conditions 4.0
%%% International de Creative Commons.
%%% http://creativecommons.org/licenses/by-sa/4.0/

\chapter*{Introduction}
\addcontentsline{toc}{chapter}{Introduction}
\markboth{Introduction}{Introduction}

Ce document est le fruit de la mise en commun d'exercices colligés au
fil du temps pour nos cours de théorie de la crédibilité à
l'Université Laval et à l'Université Concordia. Nous ne sommes
toutefois pas les uniques auteurs des exercices; certains ont, en
effet, été rédigés par les Docteurs François Dufresne et Jacques
Rioux, entre autres. Quelques exercices proviennent également
d'anciens examens de la Society of Actuaries et de la Casualty
Actuarial Society.

C'est d'ailleurs afin de ne pas usurper de droits d'auteur que ce
document est publié selon les termes du contrat Paternité-Partage des
conditions initiales à l'identique 2.5 Canada de Creative Commons. Il
s'agit donc d'un document «libre» que quiconque peut réutiliser et
modifier à sa guise, à condition que le nouveau document soit publié
avec le même contrat.

Le premier chapitre, tiré de \cite{Goulet:masters}, trace l'historique et
l'évolution de la théorie de la crédibilité, de ses origines jusqu'au
début des années 1990. Les chapitres \ref{chap:stabilite} à
\ref{chap:bstraub} --- qui correspondent aux chapitres de notre cours
--- sont quant à eux uniquement constitués d'exercices. Dans cette
seconde édition, les réponses des exercices se trouvent à la fin des
chapitres, alors que les solutions complètes se trouvent à l'annexe
\ref{chap:solutions}. De plus, à la fin de chaque chapitre
d'exercices, on trouvera une liste d'exercices suggérés dans
\cite{Klugman:lossmodels:2e:2004,Klugman:lossmodels:3e:2008}.

Un tableau synoptique des principaux résultats de crédibilité exacte
se trouve à l'annexe \ref{chap:formules}. L'annexe
\ref{chap:distributions}, présente la paramétrisation des lois de
probabilité utilisée dans les exercices. En cas de doute, le lecteur
est invité à la consulter. Il y trouvera également l'espérance, la
variance et la fonction génératrice des moments (lorsqu'elle existe)
des lois de probabilités rencontrées dans ce document. Enfin, l'annexe
\ref{chap:normale} contient un tableau des quantiles de la loi
normale

Nous tenons à remercier M.~Mathieu Pigeon pour sa précieuse
collaboration lors de la préparation de ce document, ainsi que tous
les auxiliaires d'enseignement ayant, au fil des années, contribué à
la rédaction d'exercices et de solutions.

\begin{flushright}
  Hélène Cossette \url{<helene.cossette@act.ulaval.ca>} \\
  Vincent Goulet \url{<vincent.goulet@act.ulaval.ca>} \\
  Québec, décembre 2007
\end{flushright}

%%% Local Variables:
%%% mode: latex
%%% TeX-engine: xetex
%%% TeX-master: "theorie-credibilite-avec-r"
%%% coding: utf-8
%%% End:

\tableofcontents*

%% Vignette xkcd.com
\cleartoverso
\thispagestyle{empty}
\begin{vplace}[0.45]
  \centering
  \begin{minipage}{0.9\linewidth}
    \setkeys{Gin}{width=\textwidth}
    \includegraphics{insurance.png} \\
    \footnotesize\sffamily%
    Tiré de \href{http://xkcd.com/1494/}{XKCD.com}
  \end{minipage}
\end{vplace}

\mainmatter

%%% Copyright (C) 2018 Vincent Goulet
%%%
%%% Ce fichier fait partie du projet
%%% «Théorie de la crédibilité avec R»
%%% http://github.com/vigou3/theorie-credibilite-avec-r
%%%
%%% Cette création est mise à disposition selon le contrat
%%% Attribution-Partage dans les mêmes conditions 4.0
%%% International de Creative Commons.
%%% http://creativecommons.org/licenses/by-sa/4.0/

\chapter{Introduction et perspective historique}
\label{chap:introduction-historique:introduction}

Théorie de la crédibilité: ensemble de techniques utilisées par les
actuaires pour déterminer la prime d'un assuré/contrat dans un
portefeuille hétérogène.

En faisant une tarification, un assureur cherche
\begin{enumerate}
\item d'abord à charger assez de primes pour poyer les sinistres; puis
\item distribuer équitablement ces primes entre les assurés.
\end{enumerate}

Plusieurs façons d'atteindre le second but:
\begin{enumerate}
\item d'abord par une structure de classification; puis
\item par la tarification basée sur l'expérience (\emph{experience
    rating}) $\Rightarrow$ théorie de la crédibilité.
\end{enumerate}

\begin{definition}[\emph{Experience rating}]
  La tarification basée sur l'expérience vise à assigner à chaque
  risque sa prime juste et équitable. Cette prime pour une période
  dépend exclusivement de la distribution des sinistres (inconnue) de
  ce risque pour la période. \citep{Buhlmann:credibility:1969}
\end{definition}

La tarification basée sur l'expérience exige un volume d'expérience
important. Elle est donc principalement utilisée en
\begin{itemize}
\item assurance automobile
\item accidents du travail.
\end{itemize}
Elle ne peut toutefois être utilisée, par exemple, en
\begin{itemize}
\item assurance-vie (on ne meurt qu'une fois)
\item assurance habitation (fréquence trop faible)
\end{itemize}


\section{Illustration des principes de base}
\label{sec:introduction-historique:illustration}

\begin{exemple}
  \label{ex:introduction-historique:simplifie}
  Un portefeuille d'assurance est composé de dix contrats. Les
  contrats sont à priori considérés équivalents. Les conditions
  suivantes prévalent:
  \begin{itemize}
  \item tout contrat ne peut avoir qu'au plus un sinistre par année;
  \item le montant de ce sinistre est $1$;
  \item la prime collective est $0,20$, c'est-à-dire que l'assureur
    s'attend à ce qu'en moyenne deux assurés sur dix aient un sinistre au
    cours d'une année.
  \end{itemize}


\subsection*{Situation après une année}

Voici l'expérience au sein du portefeuille après une année.

\begin{center}
  \begin{tabularx}{\tablewidth}{c*{10}{Y}}
    \toprule
    & \multicolumn{10}{c}{Contrat} \\
    \cmidrule{2-11}
    Année & 1 & 2 & 3 & 4 & 5 & 6 & 7 & 8 & 9 & 10 \\
    \midrule
    $1$ &   &   &   &   &   &   &   &   & 1 &   \\
    \bottomrule
  \end{tabularx}
\end{center}

\begin{itemize}
\item Montant de sinistre moyen par contrat $= 1/10 = 0,10$.
\item La prime collective est \emph{peut-être} trop élevée.
\item Trop peu de données pour tirer une conclusion.
\end{itemize}


\subsection*{Situation après deux années}

Après deux années, l'expérience est maintenant comme suit.

\begin{center}
  \begin{tabularx}{\tablewidth}{c*{10}{Y}}
    \toprule
    & \multicolumn{10}{c}{Contrat} \\
    \cmidrule{2-11}
    Année & 1 & 2 & 3 & 4 & 5 & 6 & 7 & 8 & 9 & 10 \\
    \midrule
    $1$ &   &   &   &   &   &   &   &   & 1 &   \\
    \midrule
    $2$ & 1 & 1 &   &   &   &   &   &   & 1 &   \\
    \bottomrule
  \end{tabularx}
\end{center}

\begin{itemize}
\item Montant de sinistre moyen par contrat $= 4/20 = 0,20$.
\item La prime collective est adéquate.
\item Le contrat $9$ a déjà eu deux sinistres.
\end{itemize}


\subsection*{Situation après dix années}

Observons maintenant la situation après dix années.

\begin{center}
  \begin{tabularx}{\tablewidth}{c*{10}{Y}}
    \toprule
    & \multicolumn{10}{c}{Contrat} \\
    \cmidrule{2-11}
    Année & 1 & 2 & 3 & 4 & 5 & 6 & 7 & 8 & 9 & 10 \\
    \midrule
    $1$ &   &   &   &   &   &   &   &   & 1 &   \\
    \midrule
    $2$ & 1 & 1 & 1 &   &   &   &   &   & 1 &   \\
    \midrule
    $3$ & 1 &   &   &   &   &   &   &   & 1 &   \\
    \midrule
    $4$ &   &   & 1 &   &   &   &   &   & 1 &   \\
    \midrule
    $5$ &   &   &   &   &   &   &   &   & 1 &   \\
    \midrule
    $6$ &   & 1 &   &   &   &   &   &   &   &   \\
    \midrule
    $7$ & 1 & 1 &   & 1 & 1 &   &   &   &   &   \\
    \midrule
    $8$ & 1 &   &   & 1 &   & 1 &   &   & 1 &   \\
    \midrule
    $9$ & 1 &   &   &   & 1 &   &   &   &   &   \\
    \midrule
    $10$ & 1 &   &   &   &   &   &   &   & 1 &   \\
    \midrule
    $\bar{S}_i$ & 0,6 & 0,3 & 0,2 & 0,2 & 0,2 & 0,1 & 0 & 0 & 0,7 & 0 \\
    \midrule
    $\bar{S}$ & \multicolumn{10}{c}{0,23} \\
    \bottomrule
  \end{tabularx}
\end{center}

\begin{itemize}
\item Montant de sinistre moyen par contrat $= 23/100 = 0,23$.
\item La prime collective est raisonnablement adéquate.
\item Le contrat $9$, avec ses sept sinistres, est en effet plus risqué.
\item Les contrats $7$, $8$ et $10$ n'ont eu aucun sinistre.
\end{itemize}



\subsection*{Conclusions}

\begin{itemize}
\item La prime collective, si elle est globalement \emph{adéquate},
  n'est en revanche clairement pas \emph{équitable}.
\item Contrairement à l'hypothèse de départ de l'assureur, le
  portefeuille est \emph{hétérogène}.
\item Besoin d'une technique de tarification basée sur l'expérience
  (\emph{experience rating}) pour adéquatement distribuer les primes
  entre les assurés.
\end{itemize}
  \qed
\end{exemple}

Deux grandes branches en théorie de la crédibilité:
\begin{enumerate}
\item Credibilité de stabilité, ou américaine, \emph{limited
    fluctuations}.

  L'assureur tient compte de l'expérience individuelle seulement si
  celle-ci est suffisamment stable dans le temps.

\item Crédibilité de précision, ou européenne, \emph{greatest
    accuracy}.

  L'assureur tient compte de l'expérience individuelle de façon à
  obtenir la meilleure estimation de l'expérience future. Le poids de
  l'expérience individuelle augmente avec l'hétérogénéité du
  portefeuille.
\end{enumerate}


\section{Historique et évolution de la théorie de la crédibilité}
\label{sec:introduction-historique:historique}

\begin{quote}
  \itshape Cette section reprend une partie du chapitre 1 de
  \cite{Goulet:masters}.
\end{quote}

La personne qui mentionne qu'elle étudie «la crédibilité» obtient
habituellement de grands yeux incrédules et interrogateurs comme seule
réponse de la part de son interlocuteur, fut-il même parfois actuaire.
Pour l'actuaire la crédibilité n'est souvent qu'un vague concept parmi
tant d'autres rencontrés au fil des examens professionnels, concept
qui eut tôt fait de fuir à toutes jambes sa mémoire une fois l'examen
réussi. Quant au non actuaire, l'étude de la crédibilité ne lui
apparaît d'aucun intérêt puisqu'il n'applique habituellement la notion
qu'aux personnes physiques: le juge est crédible lorsqu'il interprète
la loi, certains critiques de cinéma sont crédibles alors que d'autres
le sont moins, nul politicien n'est considéré crédible... Pourtant, la
définition que donne le \emph{Petit Robert} du mot crédibilité, «ce
qui fait qu'une chose mérite d'être crue; caractère de ce qui est
croyable», est assez vaste pour laisser place à d'autres
interprétations.

La population accorde donc de la crédibilité aux gens, et ce selon
leur notoriété, leurs réalisations, du fait d'expériences concluantes,
de leur honnêteté et leur droiture reconnues ou tout simplement par
affinité. En fin de compte, une personne est crédible à nos yeux s'il
est \emph{probable} que ce qu'elle dit ou fait se réalise ou soit un
succès. Par exemple, il est fort probable que l'interprétation que le
juge fait de la loi soit la bonne; ou j'irai voir le nouveau film à
l'affiche louangé par tel critique car je sais qu'il est probable que
je l'aime moi aussi.

On voit donc qu'il existe une étroite relation entre les notions de
«probabilité» et de «crédibilité». C'est pour cette raison qu'avant
d'aborder l'historique de la théorie actuarielle de la crédibilité, il
s'avérera intéressant d'étudier l'évolution de l'approche probabiliste
privilégiée en actuariat, l'approche bayésienne. Avec en plus l'étude
des réflexions sur le sujet du philosophe anglais Bertrand Russell le
lien entre les deux notions se clarifiera et la perspective historique
n'en sera que meilleure.

\subsection{L'émergence de l'approche bayésienne}
\label{sec:introduction-historique:historique:emergence}

Bien que Thomas Bayes eut présenté son fameux théorème dans un essai
en 1763, ce n'est que dans la seconde moitié des années 1950 que la
philosophie probabiliste basée sur ce théorème gagne suffisamment
d'adeptes, et donc de popularité, pour aspirer au respect. L'approche
classique monopolisait auparavant la foi des probabilistes et
statisticiens, et c'est encore aujourd'hui la philosophie enseignée en
premier lieu dans les écoles.

Le statisticien classique%
\footnote{Souvent appelé \emph{objectivist} en anglais. L'expression
  est toutefois difficile à traduire adéquatement en français et c'est
  pourquoi usage on utilisera le terme «classique»} %
a une vision très stricte, objective et mathématique de la probabilité
et de l'usage qu'il est possible d'en faire pour inférer un résultat à
partir de données. Pour lui, une probabilité est essentiellement la
limite d'une série de fréquences relatives où la symétrie joue un rôle
primordial: si un événement peut connaître $n$ réalisations
différentes mutuellement exclusives et ayant une chance égale de se
réaliser, et si de ces réalisations $m$ ont l'attribut $A$, alors la
probabilité de $A$ est $m/n$. Des énoncés du type «il est probable que
le Québec devienne indépendant au cours des deux prochaines années» ou
«il est probable qu'il neige demain» n'ont donc aucun sens à ses yeux.
D'ailleurs, les statisticiens classiques croient que leur travail se
limite exclusivement à l'analyse des données et qu'il ne leur
appartient pas de prendre des décisions à la lumière des résultats
obtenus. Il sera expliqué plus loin que, par opposition, les bayésiens
voient en la prise de décision le but de tout travail statistique.

Un autre élément majeur qui caractérise l'approche classique --- et
c'est celui qui concerne particulièrement la théorie de la crédibilité
--- est l'absence de prise en compte de toute information à priori
relative au problème étudié. On préfère «laisser les données parler
d'elles-mêmes». C'est ce qui fut à l'origine de la contestation devant
mener à l'émergence de l'approche bayésienne.

Leonard J.\ Savage publie en 1954 \citep{Savage:foundations:1954} un des premiers
livres consacrés à l'étude de la statistique dans une optique
bayésienne. Il défend alors la vision individualiste de la probabilité
(\emph{personalistic view of probability}) qu'il qualifie de «seul
concept de probabilité». Comme son nom l'indique, cette vision tâche
de se rapprocher le plus possible du raisonnement humain.  La
probabilité devient donc un indicateur de l'opinion d'une personne à
propos d'un événement et, puisqu'une opinion est --- habituellement!
--- sujette à changement suite à l'ajout d'informations, l'inférence
statistique à partir de nouvelles données constitue alors le mécanisme
de révision de l'opinion. De plus, l'adhérant à cette vision refuse
obstinément de rejeter toute information parallèle ou à priori au
problème sous prétexte que, justement, l'humain tire profit de ces
informations lorsqu'il prend une décision. Si la question «quel degré
de conviction puis-je atteindre à la suite de l'ajout de cette information?»
est hors du champ de la statistique pour le statisticien classique,
elle est au contraire au c{\oe}ur même du problème pour
l'individualiste.  Ce dernier répond d'ailleurs à la question en
utilisant le théorème de Bayes --- d'où le qualificatif de bayésien.
Cet algorithme donne la nouvelle probabilité d'un événement $A$ à
partir de sa probabilité originale et des nouveaux faits marquants
venant s'y rattacher. Ainsi, si $B$ constitue l'ensemble des nouveaux
faits, alors
\begin{equation*}
  P[A|B]= \frac{P[B|A]P[A]}{P[B]},
\end{equation*}
où $P[A]$ est la probabilité à priori que l'événement $A$ se réalise.

De plus, les bayésiens évitent l'apparente contradiction entre
l'objectivité scientifique et l'irrationalité humaine en postulant un
individu idéal, conséquent dans ses décisions. Par exemple, cet
individu, lorsque confronté à un choix, choisira toujours l'option la
plus probable.

Savage donne dans son ouvrage \emph{The foundations of statistics}
quelques expressions synonymes pour \emph{personal probability}:
\emph{subjective probability}, \emph{psychological probability} et,
finalement, \emph{degree of conviction}. Le lecteur est invité à
conserver tout particulièrement en mémoire la dernière expression. En
effet, la section suivante étudie sommairement les travaux du
philosophe Bertrand Russell et il sera intéressant de constater à quel
point les travaux de deux disciplines différentes ont su converger.

\subsubsection{Les réflexions de Russell}
\label{sec:introduction-historique:russell}

Lord Bertrand Russell (1872-1970) est un éminent philosophe et
logicien britannique, pacifiste invétéré et dont l'activité la plus
marquante se rapporta aux mathématiques et à la logique. Il est entre
autres le fondateur du logicisme, doctrine selon laquelle «les
mathématiques seraient soumises à la formalisation de la logique et
s'y réduiraient» (\emph{Le Petit Larousse}). Il fut récipiendaire du
Prix Nobel de littérature en 1950.

Joseph Butler (1692--1752) disait: «la probabilité est le guide de la
vie d'une personne».  En effet, il est raisonnable de croire que
lorsque deux événements ou plus ont des chances de se réaliser, la
décision d'une personne sera basée sur celui qui se réalisera avec la
plus forte probabilité. Mais ce type de probabilité est-il le même que
dans l'énoncé: «la probabilité d'obtenir un double six au lancer de
deux dés est de un sur trente-six»? Russell soutient que non. Voici
son raisonnement, que l'ont peut retrouver de façon plus exhaustive
dans son ouvrage de 1948 \citep{Russell:human:1948}.

De toute évidence, cette dernière est la probabilité dite
\emph{mathématique}, obéissant aux axiomes de la théorie des
probabilités, que l'on peut grossièrement réduire au quotient de deux
nombres: le cardinal d'une classe spécifique (double six) et celui
d'une classe fondamentale (les réalisations possibles du lancer de
deux dés). Elle s'attarde aux énoncés d'ordre général, se référant à
des classes «anonymes». Un autre exemple serait, en assurance-vie: «la
probabilité qu'un homme non-fumeur âgé de 30 ans décède dans l'année
qui vient est, selon la table CSO 1958, de $0,00213$». Jamais dans ces
deux exemples il n'est fait référence à une personne ou un objet en
particulier.

Cependant, en passant à un cas particulier du genre «la probabilité
que je vive jusqu'à 90 ans est de $60~\%$», est-on toujours dans le
domaine de la probabilité mathématique? Au passage d'un cas général à
un cas particulier (et en ne considérant pas le cas particulier comme
une simple réalisation du phénomène général), une personne devrait
être en mesure d'intégrer \emph{tous} les éléments d'importance à sa
prise de décision. Cette masse d'information lui permettrait alors de
savoir avec certitude si oui ou non sa longévité atteindra les 90 ans.
Et dès lors elle se retrouverait hors du champ de la probabilité.

Bien entendu, il est pour ainsi dire impossible d'atteindre un tel
niveau de certitude. C'est pourquoi la probabilité demeure le guide de
la vie de la personne, parce que \emph{son savoir s'avère
  insuffisant}. Mais encore, si ce guide est la probabilité
mathématique, il doit être possible de calculer et d'identifier
\emph{la} probabilité (c'est-à-dire \emph{la} classe fondamentale),
sinon le guide fait faux bon à son utilisateur! La personne en quête
de cette probabilité évaluera donc d'abord elle-même ses chances de
vivre jusqu'à 90 ans, puis ira probablement consulter un médecin, ou
même une voyante, qui chacun lui donnera son avis, pas nécessairement
objectif. Cette personne ne pourra que croire en totalité ou en partie
les divers jugements colligés, aussi lorsqu'il affirmera «la
probabilité que je vive jusqu'à 90 ans est de $60~\%$», il attribuera
en réalité une \emph{crédibilité} de $60~\%$ à l'expression «je vais
vivre jusqu'à 90 ans». C'est le mieux qu'il pourra faire, tout le
savoir de l'humain comportant une part de doute ou, à
l'inverse, seulement un \emph{degré de crédibilité}.

Crédibilité, voilà donc le nom qu'attribue Russell à ce second type de
probabilité, la distinction entre la probabilité mathématique et la
crédibilité se faisait au \emph{passage du général au particulier}.
Une relation subsiste néanmoins entre les deux notions, et c'est que
la probabilité mathématique est un instrument de mesure de la
crédibilité. De plus, toujours selon Russell, tout énoncé comporte un
degré de crédibilité intrinsèque, si minime soit-il, car une personne
a toujours un certain bagage de connaissances lui permettant de se
faire une opinion à priori. Finalement, la crédibilité peut être
augmentée par l'ajout d'information, le gain marginal allant en
décroissant.

Cette théorie de Bertrand Russell date de 1948 (ou du moins sa
publication). L'énoncé sommaire ci-dessus suffit pour constater à quel
point cette théorie est près de l'approche bayésienne en statistique
qui commençait alors à sortir de l'anonymat. Cela n'est pas surprenant
si l'on se rappelle que les fondateurs de l'école bayésienne ont,
comme des philosophes le font, basé leur théorie sur le comportement
humain. Ainsi, ce que Russell nomme probabilité mathématique correspond
en fait à l'approche classique en statistique et il ne la considère
pas comme un fin en soi, mais bien comme un outil pour une prise de
décision. C'est cette prise de décision, via la détermination d'un
degré de crédibilité, qui constitue la fin de l'utilisation de la
probabilité (ou de la statistique). D'ailleurs, l'expression
\emph{degree of conviction} proposée par Savage n'est-elle pas
synonyme de celle de Russell, degré de crédibilité?

Cependant, là où le philosophe fortifie le plus la thèse bayésienne,
c'est lorsqu'il soutient que tout énoncé comporte un degré de
crédibilité intrinsèque car, ce faisant, il défend l'apport non
négligeable de l'information à priori.

\subsection{Et les actuaires dans tout ça?}
\label{sec:introduction-historique:actuaires}

Longtemps avant l'apparition de l'approche bayésienne ou même les
théories de Russell, l'information à priori était prise en compte par
les actuaires. En effet, lorsqu'ils évaluaient une prime, jamais ils
n'auraient considéré ne \emph{rien} connaître du risque. À partir de
différents critères, ils arrivaient au moins à le regrouper avec
d'autres risques semblables. Toutefois, le mécanisme de prise en
compte de cette information était souvent ad hoc et mal compris à
la fois par les actuaires et les statisticiens. C'était la
crédibilité.

Si la crédibilité a su se développer et faire le bonheur de ses
utilisateurs pendant de nombreuses années sans les bases statistiques
de l'approche bayésienne, il n'en demeure pas moins que l'arrivée de
celle-ci marqua un virage important dans l'évolution de la
crédibilité, contribuant d'ailleurs à en faire davantage une théorie.
Les lignes suivantes tâcheront de retracer l'évolution historique de
la théorie de la crédibilité, de ses balbutiements aux débuts du
siècle à son adolescence actuelle, tout en faisant le lien avec les
théories présentées plus haut.

La petite histoire suivante, inventée par le Dr François Dufresne pour
un cours d'introduction à la théorie de la crédibilité à l'Université
Laval, illustre sans doute bien comment cette théorie a pu voir le
jour.

Vers les années 1910, aux États-Unis, la multinationale General Motors
et le petit constructeur indépendant Tucker sont assurés chez Allstate
contre les accidents de travail (\emph{workers compensation}), avec
quelques autres fabricants d'automobiles. Un taux moyen est calculé à
partir de l'expérience de l'ensemble de ces fabricants et c'est ce
taux qui est chargé à chacun. Or, la GM calcule elle-même son taux et
s'aperçoit qu'il serait, année après année, inférieur à celui qu'on
lui charge et ce, grâce à une expérience meilleure que celle du
groupe. Exaspérée par une telle situation, elle demande à Allstate
qu'on lui charge son propre taux, ceci sous prétexte que son nombre
d'employés important est un gage de stabilité de l'expérience entre
les années.

Les actuaires de Allstate sont intuitivement d'accord avec
l'argumentation de GM, aussi s'apprêtent-ils à accéder à sa demande.
Un petit problème les fait cependant hésiter: si le nombre d'employés
de GM est clairement assez gros pour que l'on se fie à son expérience
et celui de Tucker trop petit pour faire de même, où fixera-t-on la
limite entre un nombre d'employés fiable et un non fiable?

Il est généralement reconnu que le premier actuaire à avoir proposé
une solution au problème dans la littérature est Arthur H.\ Mowbray,
dans le premier numéro des Proceedings de la Casualty Actuarial
Society, en 1914 \citep{Mowbray:1914}. En supposant connue la
probabilité $q$ qu'un accident survienne, il désire calculer le nombre
minimal d'employés assurés $n$ de telle sorte que la probabilité que
le nombre d'accidents ne s'éloigne pas de plus de $100k$~\% de la
moyenne (ou du mode%
\footnote{Mowbray privilégie la valeur la plus fréquente, le mode,
  mais s'en remet à la moyenne par simplicité et sous prétexte que ces
  deux valeurs sont presque égales.}) %
soit supérieure à $100p$~\%. Si l'on note $N$ la distribution du
nombre d'accidents, alors l'énoncé précédent s'écrit en langage
mathématique
\begin{equation*}
  P \left[ (1 - k)E[N] \leq N \leq (1 + k)E[N] \right] \geq p,
\end{equation*}
où $N \sim \text{Binomiale}(n, q)$. Par la suite, l'utilisation de
l'approximation normale pour la distribution de $N$ permet d'éviter
l'arbitrage entre le mode et la moyenne et d'obtenir aisément une
réponse.

Les bases de la crédibilité de stabilité, ou \emph{limited
  fluctuations}, étaient dès lors posées.

La solution de Mowbray était intéressante car appuyée par des
arguments probabilistes. Elle connut d'ailleurs de multiples
adaptations qui lui permirent de rester d'actualité jusqu'à nos jours.
Moyennant la détermination d'une distribution pour le nombre de
sinistres, les praticiens pouvaient désormais résoudre par un calcul
simple le problème auquel ils faisaient face: fixer un seuil
d'admissibilité à la pleine crédibilité.

Les assurés eurent cependant tôt fait de soumettre un autre problème
aux actuaires. Telle que présentée jusqu'à maintenant, la théorie de
la crédibilité n'admettait que deux niveaux: $0$ et $1$. Or, cette
situation pouvait se traduire, pour un employeur situé tout juste sous
le seuil d'admissibilité, en une différence significative dans la
prime à payer. De plus, la notion même de pleine crédibilité ne
ralliait pas l'ensemble des actuaires, certains croyant que jamais les
données ne sont fiables à $100~\%$.

C'est pour répondre à ces critiques que fut introduit le concept de
crédibilité partielle, dont on attribue la première véritable théorie
sur le sujet à \cite{Whitney:1918}. Dès 1918 donc, Whitney mentionne la
«nécessité, par souci d'équité pour l'assuré, de pondérer d'un côté
l'expérience collective et de l'autre l'expérience individuelle».
L'essence de la crédibilité consistait à calculer cette pondération.

À au moins deux reprises dans l'histoire, la crédibilité de précision
(\emph{greatest accuracy}) a eu la chance de s'imposer avant la
contribution de Bühlmann en 1967. La première se trouve dans les
travaux de Whitney. Dans son article, ce dernier retient quatre
éléments qui influenceront la pondération à donner à l'expérience
individuelle: l'exposition, le niveau de risque, la crédibilité de la
prime collective et l'homogénéité du groupe. D'ailleurs, il mentionne:

\begin{quote}
  «Il n'y aurait pas de problème d'\emph{experience rating} si chaque
  risque dans le groupe était typique du groupe, car dans ce cas les
  variations dans l'expérience ne seraient que purement aléatoires.»
  (Traduction libre)
\end{quote}

Or, la notion d'homogénéité du collectif est au c{\oe}ur même de la
crédibilité de précision. Whitney modélise l'hétérogénéité du
collectif en supposant que les moyennes des divers risques sont
distribuées selon une loi normale. De là, par un développement
mathématique lourd et laborieux, il obtient une expression pour la
prime individuelle ($P$) de la forme
\begin{equation*}
  P = z X + (1 - z) C,
\end{equation*}
où $X$ est l'expérience individuelle et $C$, l'expérience collective.
Ces valeurs sont pondérées par le facteur de crédibilité, $z$, dont
Whitney obtient une formule de la forme $n/(n + K)$. $K$ n'est alors
pas une constante arbitraire, mais bien une expression explicite
dépendant des divers paramètres du modèle. Par souci de simplicité,
Whitney suggère cependant de fixer $K$ comme une constante à
déterminer au jugement de manière à éviter les trop grandes
fluctuations entre la prime individuelle et la prime collective. Ce
faisant, il s'écarte d'une conception de la crédibilité visant la
précision pour encourager plutôt celle visant la stabilité. La
crédibilité de précision meurt dans l'{\oe}uf à cause de
considérations d'ordre pratique.

Whitney fut de plus critiqué par \cite{Fisher:Whitney:discussion:1919}
pour avoir utilisé une méthode jugée hérétique à l'époque en
statistique, la règle de Bayes. Fisher cite d'ailleurs quelques
personnes faisant autorité s'étant prononcées contre l'usage de cette
règle. En fait, Whitney utilise la version la plus simple de la règle,
celle même proposée par Bayes\footnote{%
  Selon \cite{Bailey:1950}.}, %
qui suppose qu'à priori tous les événements ont une chance égale de se
réaliser. Cette modélisation était appelée par ses adeptes le
«Principe de la raison insuffisante» et par ses détracteurs
«L'hypothèse de distribution uniforme de l'ignorance»%
\footnote{\emph{Principe of insufficient reason} et \emph{Assumption
    of the equal distribution of ignorance.}}. %
Dans le développement de Whitney, cela revient à supposer que toutes
les valeurs possibles pour la moyenne du groupe sont équiprobables.
Les réserves de Fisher sur ce point étaient donc justifiées...

Il est difficile aujourd'hui de mesurer l'impact que la volte-face de
Whitney en faveur des arguments de stabilité a pu avoir sur la
pratique de la crédibilité. Toujours est-il que les actuaires ont
utilisé intensivement pendant un demi-siècle la forme $z = n/(n + K)$ pour
calculer des facteurs de crédibilité basés essentiellement sur la
stabilité. Encore aujourd'hui, c'est l'approche privilégiée aux
États-Unis. Pourtant, si les pratiques se sont quelque peu figées, la
théorie, elle, n'a pas cessé d'évoluer.

L'approche de précision tente à nouveau sa chance par l'entremise de
Arthur L.~Bailey, et ce à deux reprises (\cite{Bailey:1945} et
\cite{Bailey:1950}). En 1945 d'abord, Bailey obtient une expression pour
la crédibilité dans ce qui semble être l'univers non paramétrique
exploré plus tard par Bühlmann. Seulement, une notation confuse rend
le texte difficilement déchiffrable et le condamne à la marginalité.

L'article de 1950 avait, lui, davantage le potentiel pour ébranler les
acquis en théorie de la crédibilité. Bailey débute son article en
relevant les confrontations historiques entre statisticiens et
actuaires au sujet de la crédibilité. On apprend alors qu'il y a plus
de trente ans, la polémique au sujet de la crédibilité était la même
que de nos jours: les statisticiens «purs» crient au scandale,
estimant les méthodes actuarielles contraires à toute théorie
statistique, et les actuaires leur répondent que «peut-être, mais ça
fonctionne!» Le point majeur de discorde est l'utilisation par les
actuaires de la règle de Bayes. Ainsi, \cite{Bailey:1950} écrit:

\begin{quote}
  «Présentement, presque toutes les méthodes d'estimation présentées
  dans les livres de méthode statistiques ou enseignées dans les
  universités américaines sont basées sur un équivalent de l'hypothèse
  selon laquelle toute information parallèle ou connaissance à priori
  est inutile. (...) Des philosophes [Russell] ont récemment étudié la
  crédibilité à accorder à divers éléments de savoir, remettant par
  conséquent en doute la philosophie adoptée par les statisticiens.
  Par contre, il semble que ce ne soit que dans le domaine de
  l'actuariat qu'on ait assisté à une réelle révolte contre la mise de
  côté de tout savoir à priori au moment d'estimer une quantité à
  l'aide de nouvelles données.» (Traduction libre)
\end{quote}

Bailey se prononce donc clairement en faveur de la philosophie
bayésienne. Il semble même qu'il fut le premier à en défendre si
énergiquement la pertinence dans le processus de tarification. Mais
ceci, fait-il remarquer, à condition d'utiliser une généralisation de
la règle de Bayes, faite par Laplace en 1820, de manière à éviter le
controversé «Principe de la raison insuffisante». Cette généralisation
rend la règle de Bayes applicable même si les événements n'ont pas à
priori une chance égale de se réaliser.

À partir de ces prémisses, Bailey démontre qu'en minimisant l'erreur
quadratique dans un contexte bayésien, l'estimateur obtenu est une
fonction linéaire des observations%
\footnote{De la forme $c_0 + \sum_{i,j} c_{ij} X_{ij}$.} %
qui correspond exactement à la prime de crédibilité, et ce pour les
combinaisons de distributions binomiale/bêta, Poisson/gamma et
normale/normale. Il est possible d'en déduire un facteur de
crédibilité encore une fois de la forme $z = n/(n + K)$ où $K$ est une
combinaison des paramètres du modèle.  Contrairement à Whitney, Bailey
ne propose pas d'évaluer $K$ au jugement, mais bien de s'en tenir à
l'expression développée algébriquement.

De plus, conscient que la procédure utilisée est une modélisation de
l'hétérogénéité d'un groupe, Bailey mentionne pour la première fois
que la crédibilité pourrait être utilisée hors du domaine de
l'assurance générale. Malheureusement pour Bailey, le manque de
popularité de l'approche bayésienne au sein de la communauté
statistique au moment de la parution reléguera son article dans
l'ombre. Sans doute l'approche très «théorique» de l'article, publié
dans une revue essentiellement de praticiens, aura-t-elle aussi
contribué à cet état de fait. Pourtant, dans un avenir pas si
lointain, les travaux de Bailey seront reconnus pour leur importance
et leur avant-gardisme. Cette reconnaissance proviendra cependant
majoritairement d'outre-Atlantique.

L'apport de Bailey à la théorie de la crédibilité peut être résumé en
deux points principaux: l'introduction explicite du principe de Bayes
dans le processus de tarification et la découverte de la linéarité de
l'estimateur bayésien sous certaines conditions%
\footnote{Sur ce point toutefois, il est difficile de savoir qui fut
  le pionnier. \cite{Norberg:credibility:1979} relève que
  \cite{Keffer:1929} a obtenu le même résultat pour le modèle
  Poisson/gamma. Toujours selon Norberg, il existerait aussi des
  références antérieures à 1929.} %
Ces développements parvinrent aux oreilles de la communauté
actuarielle européenne par la bouche de Bruno de~Finetti au colloque
ASTIN à Trieste, en 1963. Depuis quelques années déjà, les
chercheurs européens tâchaient de trouver une justification théorique
à ces formules de crédibilité américaines qui fonctionnaient si bien.
L'approche bayésienne avait là aussi fait des progrès importants,
notamment grâce à Bruno de~Finetti dans les années 1930 et Ove Lundberg
dans les années 1940.

Parallèlement à tout cela, une nouvelle branche de la théorie
bayésienne voyait le jour sous l'initiative de Herbert Robbins:
l'approche bayésienne empirique. Celle-ci sera d'une importance
capitale dans le développement futur de la crédibilité de précision
puisqu'elle lui permettra de sauter le mur entre la théorie et la
pratique. L'approche bayésienne empirique
\citep{Robbins:empiricalbayes:1955,Robbins:empiricalbayes:1964}
s'attaque au principal problème pratique de l'approche bayésienne
pure, soit la fréquente ignorance de la distribution à priori.
Jusqu'alors, lorsqu'une telle situation se présentait, il était
coutume d'éviter le problème en réfutant l'aspect aléatoire d'une
partie de la décision \citep{Neyman:1962}. Grossièrement, la thèse de
Robbins consiste à supposer que, bien qu'elle soit inconnue, la
distribution à priori existe et qu'il est possible de l'estimer --- ne
serait-ce qu'indirectement --- à partir de données issues de plusieurs
expériences similaires. Selon lui \citep{Robbins:empiricalbayes:1964}:
\begin{quote}
  «L'approche bayésienne empirique de problèmes statistiques de prise
  de décision est applicable lorsque la même décision se présente à
  répétition et indépendamment avec une distribution fixe, mais
  inconnue du paramètre.» (Traduction libre)
\end{quote}

Comme devait plus tard le mentionner Bühlmann, cela cadre
admirablement bien avec le problème de l'\emph{experience rating}.

Tous ces éléments --- les travaux de Bailey, la popularité
grandissante des approches bayésienne et bayésienne empirique ---
étaient réunis lors du congrès ASTIN de 1965, à Lucerne. Hans
Bühlmann y redéfinit alors le problème fondamental en \emph{experience
  rating} et présente la solution qui allait révolutionner la théorie
de la crédibilité. En forçant la prime bayésienne à être linéaire,
Bühlmann (1967, 1969) obtient, dans un cadre non paramétrique, un
facteur de crédibilité de la forme $z = n/(n+K)$, avec une expression
simple et générale pour $K$. Le virage sera alors définitivement pris
en faveur de l'approche de précision et l'essentiel de la recherche se
fera en Europe. C'est pourquoi l'approche traitant l'hétérogénéité est
aujourd'hui souvent appelée «crédibilité européenne» malgré qu'elle
origine des États-Unis (notamment dans les travaux de Bailey).
L'approche de stabilité, \emph{limited fluctuations}, a par conséquent
reçu l'appellation «crédibilité américaine».

On peut sans crainte poser que le célèbre modèle de Bühlmann
marque le début de l'histoire contemporaine de la théorie de la
crédibilité. Celle-ci étant bien davantage documentée et connue, on va
maintenant se contenter de présenter chronologiquement les principaux
modèles qui suivirent celui de Bühlmann%
\footnote{La liste ne comporte que les principaux modèles et n'est
  donc en rien exhaustive.}. %
La plupart de ces modèles se veulent des généralisations, de plus en
plus poussées, du modèle original de Bühlmann et bon nombre d'entres
eux seront étudiés ultérieurement dans ce mémoire.

Le modèle de Bühlmann se décompose lui-même en deux parties
communément appelées «modèle original» et «modèle classique». Le
premier pose les bases de la nouvelle théorie, tandis que l'apport de
la théorie bayésienne empirique fait du second un modèle plus
pratique. La première généralisation de ces modèles voit le jour en
1970, alors que Bühlmann s'adjoint son étudiant au doctorat Erwin
Straub pour développer le très célèbre modèle qui portera leurs noms
\citep{BuhlmannStraub:1970}. L'ajout de poids aux données et la
définition d'estimateurs des paramètres de structure constituent les
principales améliorations au modèle de Bühlmann. Elles sont toutefois
de taille et permettront à la crédibilité de précision de
véritablement faire une percée dans la pratique de l'assurance. Le
modèle de Bühlmann--Straub constitue encore aujourd'hui un standard et
est couramment utilisé dans les compagnies d'assurance, en Europe
surtout.

En 1974, Jewell fait la première de ses deux plus grandes
contributions au développement de la théorie de la crédibilité. Il
démontre \citep{Jewell:exact:1974} que l'estimateur bayésien est
linéaire pour toute fonction de vraisemblance membre de la famille
exponentielle utilisée avec sa conjuguée naturelle. Ce faisant, Jewell
ne fait que confirmer certains des résultats obtenus avant lui par
\cite{Bailey:1950} et \cite{Mayerson:bayesian:1964}, mais les unifie
en une formulation générale.

Les deux années suivantes furent fastes pour la théorie de la
crédibilité. D'abord, \cite{Hachemeister:1975} généralise le modèle de
Bühlmann--Straub en incorporant la régression linéaire à la théorie de
la crédibilité de précision. Plus tard, on poussera même cette idée un
peu plus loin en étudiant la régression non-linéaire
\citep{DeVylder:non-linear:1985}. Puis, non satisfait de la manière
qu'a le modèle de Bühlmann--Straub d'intégrer à la prime les données
collatérales, \cite{Jewell:hierarchical:1975} présente sa solution: la
crédibilité hiérarchique. Ce qui est maintenant appelé le modèle
hiérarchique de Jewell constitue une importante généralisation de
plusieurs modèles de crédibilité. Toujours en 1975,
\cite{Gerber:updating:1975} définissent les propriétés et les
conditions menant à des primes de crédibilité de type «mise à jour»
(\emph{updating type}). En 1976, \citep{DeVylder:semilinear:1976}
présente ses modèles de crédibilité semi-linéaire et semi-linéaire
optimal. La même année, il propose une formulation du problème de la
crédibilité en termes d'espaces de Hilbert
\citep{DeVylder:geometrical:1976}. Norberg et
\cite{Taylor:abstractcredibility:1977} se sont également intéressés à
l'étude de la théorie de la crédibilité dans le cadre abstrait des
espaces de Hilbert. \cite{Norberg:credibility:1979} a publié un imposant
article révisant l'essentiel de la théorie de la crédibilité connu
jusqu'alors. Il s'agit, encore aujourd'hui, d'une référence de choix
pour qui désire approfondir des connaissances de base en théorie de la
crédibilité.

Si les années 1970 furent celles de l'apparition de nombreux modèles de
crédibilité de précision, les années 1980 furent plutôt celles de
l'étude des estimateurs des paramètres de structure.
\cite{DeVylder:estimation:1978,DeVylder:estimation:1981,DeVylder:estimation:PCiI:1984}
s'est alors avéré un acteur important, de même que
\cite{Norberg:empiricalbayes:1980}, \cite{Gisler:trimming:1980} et
\cite{Dubey:estimation:1981}. On doit à ces derniers la première étude
exhaustive des propriétés asymptotiques des estimateurs de variance
dans le modèle de Bühlmann--Straub. La découverte de nouveaux
estimateurs et leur maîtrise va permettre la plus large diffusion de
la crédibilité de précision. Des volumes sur le sujet sont publiés
\citep{Goovaerts:EAM:1990} dont certains
\citep{Goovaerts:credibility:1987} sont expressément destinés aux
actuaires {\oe}uvrant dans les compagnies d'assurance. De nombreux
articles importants en théorie de la crédibilité sont publiés par des
chercheurs au service de compagnies d'assurance: mentionnons Straub,
Sundt, Dubey, Gisler, Reinhard ou même Goovaerts, à titre de
consultant. Cela contribue donc à populariser l'utilisation de la
théorie dans des applications pratiques.

La période s'étendant du milieu des années 1980 jusqu'au début des
années 1990 a été marquée par un ralentissement dans la recherche en
théorie de la crédibilité. Au cours des dernières années cependant, on
note un regain d'intérêt pour la recherche d'estimateurs des
paramètres de structure plus efficaces et dont les propriétés seraient
davantage connues. Ainsi, \cite{Kunsch:robust:1992} et
\cite{Gisler:robust:1993} introduisent la statistique robuste en
théorie de la crédibilité, ce qui ouvre par le fait même un nouveau et
vaste champ de recherche. De leur côté, De~Vylder et Goovaerts,
toujours très actifs dans le domaine, proposent dans une série
d'articles des estimateurs optimaux sous certaines conditions
\citep{DeVylder:zeroexcess:summary:1991,%
  DeVylder:zeroexcess:classical:1992,%
  DeVylder:zeroexcess:BS:1992}. Des recherches sont présentement en
cours desquelles, nous promet-on, devraient ressortir de toutes
nouvelles façons de trouver des estimateurs pour les principaux
modèles de crédibilité.

La grande majorité des intéressés à la théorie de la crédibilité se
rallient au modèle de base proposé par Bühlmann de même qu'à ses
extensions. De plus, l'historique tracé ci-dessus correspondant à ce
qui est généralement admis dans le domaine. À une exception près
semble-t-il: \cite{Zehnwirth:studyguide:1991} qui, sans renier les
travaux de Bühlmann, a une manière bien à lui de concevoir et
présenter la théorie de la crédibilité de précision. Selon lui, les
formules de crédibilité sont si étroitement liées à la régression
linéaires qu'elles n'en seraient que de simples dérivés. Karl
Friedrick Gauss étant à l'origine de certains types de formules de
régression, Zehnwirth en arrive à la conclusion suivante: «cela
signifie que Gauss~(1795) a dérivé la plupart des formules de
crédibilité».

\subsection{En conclusion}

Ce chapitre s'est ouvert sur une présentation de l'approche bayésienne
en statistiques. Celle-ci allait en quelque sorte devenir le
dénominateur commun des deux théories par la suite étudiées, celle de
Bertrand Russell et la théorie actuarielle de la crédibilité. Ce pont
permet l'intéressante comparaison de la définition donnée par Russell
au mot «crédibilité» à sa signification actuarielle. On voit ainsi
qu'en actuariat comme dans la philosophie de Russell, la crédibilité
apparaît lors du passage du général au particulier, soit lors de la
tarification d'un assuré pris isolément et non plus comme simple
membre d'un groupe. Le traitement de l'information à priori ainsi que
le gain marginal de crédibilité décroissant en théorie de la
crédibilité de stabilité ou à la crédibilité de précision. Là n'en est
pas le bus de toute façon, puisque chacune de ces deux théories est
valable en soi, seuls leurs buts étant différents.

%%% Local Variables:
%%% mode: latex
%%% TeX-engine: xetex
%%% TeX-master: "theorie-credibilite-avec-r"
%%% coding: utf-8
%%% End:

%%% Copyright (C) 2018 Vincent Goulet
%%%
%%% Ce fichier fait partie du projet
%%% «Théorie de la crédibilité avec R»
%%% http://github.com/vigou3/theorie-credibilite-avec-r
%%%
%%% Cette création est mise à disposition selon le contrat
%%% Attribution-Partage dans les mêmes conditions 4.0
%%% International de Creative Commons.
%%% http://creativecommons.org/licenses/by-sa/4.0/

\chapter{Crédibilité de stabilité}
\label{chap:stabilite}

La théorie de la crédibilité est apparue dans le domaine des accidents
du travail au début des années 1900. Les actuaires cherchaient alors à
déterminer la taille minimale (en nombre d'employés) qu'un employeur
devait atteindre pour que son expérience soit considérée pleinement
«fiable» (\emph{dependable}) ou «crédible».

Tel que relaté à la \autoref{sec:introduction-historique:historique},
\cite{Mowbray:1914} fournit une première solution. Il fournit du même
coup une définition de ce que doit être une prime pure crédible: «une
prime pour laquelle la probabilité est forte qu'elle ne diffère pas de
la vraie prime par plus d'une limite arbitraire».

En termes mathématiques, nous souhaitons que
\begin{displaymath}
  \Pr[(1 - k) \esp{S} \leq S \leq (1 + k) \esp{S}] \geq p,
\end{displaymath}
où $S$ représente l'expérience d'un contrat, sous une forme ou une
autre; $k$ est une petite valeur, habituellement autour de $5$~\%; $p$
une valeur près de $1$, habituellement $0,90$, $0,95$ ou $0,99$.


\section{Crédibilité complète}
\label{sec:stabilite:complete}

En crédibilité complète, un contrat d'assurance est considéré
\emph{crédible} si son expérience est \emph{stable}. Lorsque c'est le
cas, seule l'expérience du contrat est utilisée dans la tarification
de ce contrat.

\tipbox{La notion de portefeuille de contrats ne joue aucun rôle pour
  le moment.}

Intuitivement, la stabilité de l'expérience d'un contrat va de pair
avec sa «taille», qu'elle soit exprimée en termes de: volume de prime,
masse salariale, nombre d'employés, nombre de sinistres, nombre
d'années d'expérience, etc.

De plus, la taille du contrat est généralement liée à la fréquence des
sinistres, et non à la sévérité de ceux-ci.

\begin{definition}[Crédibilité complète]
  \label{def:stabilite:complete}
  Une crédibilité complète d'ordre $(k, p)$ est attribuée à
  l'expérience $S$ d'un contrat si les paramètres de la distribution
  de $S$ sont tels que la relation
  \begin{displaymath}
    \Pr[(1 - k) \esp{S} \leq S \leq (1 + k) \esp{S}] \geq p
  \end{displaymath}
  est vérifiée.
\end{definition}

La variable aléatoire $S$ représente la somme de l'expérience de
plusieurs risques. Par le Théorème central limite,
\begin{equation*}
  Z = \frac{S - \esp{S}}{\sqrt{\var{S}}}
  \longrightarrow
  N(0, 1)
\end{equation*}
lorsque le nombre de termes dans la somme $S$ augmente. La relation de
la \autoref{def:stabilite:complete} peut alors se réécrire
\begin{align*}
  \Pr
  \left[
    - \frac{k \esp{S}}{\sqrt{\var{S}}}
    \leq Z \leq
    \frac{k \esp{S}}{\sqrt{\var{S}}}
  \right]
  &\approx
    \Phi
    \left(
      \frac{k \esp{S}}{\sqrt{\var{S}}}
    \right) -
    \Phi
    \left(
      - \frac{k \esp{S}}{\sqrt{\var{S}}}
    \right) \\
  &=
    2 \Phi
    \left(
    \frac{k \esp{S}}{\sqrt{\var{S}}}
    \right) - 1 \geq p.
\end{align*}
L'inégalité est satisfaite dès lors que
\begin{equation*}
  \frac{k \esp{S}}{\sqrt{\var{S}}} \geq \zeta_{\varepsilon/2},
\end{equation*}
ou, de manière équivalente, lorsque
\begin{equation}
  \label{eq:stabilite:complete}
  \esp{S} \geq
  \left(
    \frac{\zeta_{\varepsilon/2}}{k}
  \right)
  \sqrt{\var{S}},
\end{equation}
où $\varepsilon = 1 - p$ et $\zeta_\alpha$ est le $100(1 -
\alpha)${\ieme} centile d'une loi normale centrée réduite.

\begin{exemple}[Binomiale pure]
  \label{exemple:stabilite:binomiale_pure}
  \cite{Mowbray:1914} recherchait le nombre minimal d'employés pour
  considérer l'expérience d'un employeur pleinement crédible. Ses
  hypothèses étaient que la variable aléatoire $S$ du nombre
  d'accidents par année pour un employeur est binomiale de paramètres
  $n$ et $\theta$, où $n$ représente le nombre d'employés et $\theta$
  la probabilité qu'un accident survienne au cours de l'année pour
  tout employé. Cette probabilité est supposée connue.

  Avec $\esp{S} = n \theta$ et $\var{S} = n \theta (1 - \theta)$ et en
  isolant $n$ dans l'inégalité \eqref{eq:stabilite:complete}, nous
  obtenons la relation suivante pour le critère de crédibilité
  complète:
  \begin{displaymath}
    n \geq
    \left(
      \frac{\zeta_{\varepsilon/2}}{k}
    \right)^2
    \frac{1 - \theta}{\theta}.
  \end{displaymath}
  \qed
\end{exemple}

\begin{exemple}[Poisson composée]
  \label{exemple:stabilite:comppois}
  La taille d'un contrat est souvent exprimée en termes du nombre
  espéré de sinistres dans une période --- typiquement une année. La
  distribution la plus populaire pour $S$ dans un tel cas est alors la
  Poisson composée, c'est-à-dire
  \begin{displaymath}
    S = X_1 + \dots + X_N
  \end{displaymath}
  où $N \sim \text{Poisson}(\lambda)$ et la distribution de $X_1, X_2,
  \dots$ est $F_X(\cdot)$. Nous savons que
  \begin{align*}
    \esp{S}
    &= \esp{N} \esp{X} \\
    &= \lambda \esp{X} \\
    \var{S}
    &= \var{N} \esp{X}^2 + \esp{N} \var{X} \\
    &= \lambda \esp{X}^2 + \lambda \var{X} \\
    &= \lambda \esp{X^2}.
  \end{align*}
  La relation \eqref{eq:stabilite:complete} devient alors, en isolant
  le nombre espéré de sinistre $\lambda$:
  \begin{equation}
    \label{eq:stabilite:complete_comppois}
    \begin{split}
      \lambda
      &\geq
      \left(
        \frac{\zeta_{\varepsilon/2}}{k}
      \right)^2
      \left(
        1 + \frac{\var{X}}{\esp{X}^2}
      \right) \\
      &=
      \left(
        \frac{\zeta_{\varepsilon/2}}{k}
      \right)^2
      (1 + \CV(X)^2),
    \end{split}
  \end{equation}
  où
  \begin{displaymath}
    \CV(X) = \frac{\sqrt{\var{X}}}{\esp{X}}
  \end{displaymath}
  est le coefficient de variation de la variable aléatoire $X$.

  Vous remarquerez que, dans l'inégalité
  \eqref{eq:stabilite:complete_comppois}, le seuil de crédibilité
  complète $\lambda$ augmente avec la variabilité des montants de
  sinistres (exprimée en fonction de coefficient de variation
  $\CV(X)$).

  Si $k = 0,05$ et $p = 0,90$, alors $\zeta_{0,05} = 1,645$ et
  \begin{displaymath}
    \lambda
    \geq
    \nombre{1082,41}
    \left(
      1 + \frac{\var{X}}{\esp{X}^2}
    \right).
  \end{displaymath}

  Si, de plus, $X$ est une variable aléatoire dégénérée (c'est-à-dire
  que $\Pr[X = M] = 1$ pour $M$ quelconque), alors
  \begin{displaymath}
    \lambda \geq \nombre{1082,41}.
  \end{displaymath}
  Ce cas revient en définitive à poser
  $S \sim \text{Poisson}(\lambda)$. %
  \qed
\end{exemple}

\tipbox{Le seuil de pleine crédibilité \nombre{1082} est un nombre
  célèbre en théorie de la crédibilité. Malheureusement, il est très
  souvent utilisé à tort et à travers et sans aucunement tenir compte
  des hypothèses qui nous ont permis d'obtenir cette valeur.}

\begin{exemple}
  \label{exemple:stabilite:nombre_annees}
  Dans les deux exemples précédents, nous avons déterminé le seuil de
  pleine crédibilité d'un contrat pour une seule période. Nous
  pourrions aussi choisir de fixer le seuil en fonction du nombre
  d'années d'expérience. Pour cela, il suffit de définir
  \begin{displaymath}
    S \equiv \bar{S} = \frac{S_1 + \dots + S_n}{n},
  \end{displaymath}
  où $S_1, \dots, S_n$ sont des variables aléatoires indépendantes et
  identiquement distribuées et $S_t$ est l'expérience de l'année
  $t = 1, \dots, n$.

  Nous avons alors
  \begin{align*}
    \esp{S} &= \esp{S_t} \\
    \var{S} &= \frac{\var{S_t}}{n},
  \end{align*}
  d'où, en remplaçant ces expressions dans l'inégalité
  \eqref{eq:stabilite:complete}, l'expérience d'un contrat est
  considérée pleinement crédible après
  \begin{equation*}
    n \geq
    \left(
      \frac{\zeta_{\varepsilon/2}}{k}
    \right)^2
    \frac{\var{S_t}}{\esp{S_t}^2}
  \end{equation*}
  périodes.
  \qed
\end{exemple}

\tipbox{La distribution de $S$ n'est en général pas symétrique. Est-il
  alors correct d'utiliser le Théorème central limite (TCL)? Le
  critère de crédibilité complète exige que la distribution de $S$
  soit très concentrée autour de sa moyenne, donc presque symétrique.
  Le TCL s'avère donc bien assez précis.}

\importantbox{Il y a en fait très peu d'applications légitimes de la
  crédibilité de stabilité. Un bon exemple, toutefois: la fixation du
  seuil d'admissibilité à un régime de tarification rétrospectif, où
  la stabilité de l'expérience joue un rôle important.}


\section{Crédibilité partielle}
\label{sec:stabilite:partielle}

La crédibilité complète ne connait que deux états: aucune crédibilité
ou crédibilité complète. Le besoin (ou la pression) de tenir compte en
partie de l'expérience individuelle d'un contrat se trouvant sous le
seuil de crédibilité complète apparait rapidement chez les assureurs.

\cite{Whitney:1918} propose de pondérer l'expérience individuelle
($S$) et la prime collective ($m$) par un \emph{facteur de
  crédibilité} ($0 \leq z \leq 1$) en une prime de la forme
\begin{displaymath}
  \pi = z S + (1 - z) m.
\end{displaymath}

Plusieurs formules différentes ont été proposées pour $z$. Soit $n_0$
le seuil de crédibilité complète. Les formules les plus populaires
sont
\begin{align*}
  z &= \min \left\{ \sqrt{\frac{n}{n_0}}, 1 \right\}, \\
  z &= \min \left\{ \left( \frac{n}{n_0} \right)^{2/3}, 1 \right\}, \\
  \intertext{et la formule de Whitney}
  z &= \frac{n}{n + K},
\end{align*}
où $K$ est une constante déterminée au jugement de façon à limiter les
fluctuations dans la prime d'une année à l'autre (\emph{swing}).

Le but de cette approche consiste à incorporer autant d'expérience que
individuelle possible dans la prime sans trop en affecter la stabilité
d'une année à l'autre.

\importantbox{La distribution des primes dans cette approche de
  crédibilité partielle est basée uniquement sur la taille des
  assurés. Rien n'assure que la tarification est précise ou
  équitable.}


\section{Exercices}
\label{sec:stabilite:exercices}

\Opensolutionfile{reponses}[reponses-stabilite]
\Opensolutionfile{solutions}[solutions-stabilite]

\begin{Filesave}{solutions}
\section*{Chapitre \ref*{chap:stabilite}}
\addcontentsline{toc}{section}{Chapitre \protect\ref*{chap:stabilite}}

\end{Filesave}

\begin{exercice}
  Le montant total des sinistres, $S$, a une distribution binomiale
  composée de paramètres $n = \nombre{1000}$ et $\theta = 0,6$. La
  distribution du montant des sinistres est une log-normale de
  paramètres $\mu = 3$ et $\sigma² = 4$. Calculer $\esp{S}$ et
  $\var{S}$.
  \begin{rep}
    $\esp{S} = \nombre{89047}$, $\var{S} = \nombre{713633042}$
  \end{rep}
  \begin{sol}
    Nous avons $S = X_1, \dots, X_N$ avec
    $N \sim \text{Binomiale}(\nombre{1000},\, 0,6)$ et
    $X \sim \text{Log-normale}(3, 4)$, d'où
    \begin{align*}
      \esp{N} &= (\nombre{1000})(0,6) = 600 \\
      \var{N} &= (\nombre{1000})(0,6)(1 - 0,6) = 240 \\
      \esp{X} &= e^{3 + 4/2} = e^5 \\
      \var{X} &= e^{2 \cdot 3 + 4}(e^4 - 1) = e^{10}(e^4 - 1) \\
      \intertext{et, par conséquent,}
      \esp{S}
      &= \esp{N} \esp{X} \\
      &= 600\, e^5 \\
      &= \nombre{89047} \\
      \var{S}
      &= \esp{N} \var{X} + \var{N} \esp{X}^2 \\
      &= 600\, e^{10}(e^4 - 1) + 240\, e^{10} \\
      &= \nombre{713633042}.
    \end{align*}
  \end{sol}
\end{exercice}

\begin{exercice}
  Sachant que le nombre de sinistres a une distribution binomiale
  négative de paramètres $n = 4$ et $\theta = 0,5$ et que la sévérité
  d'un sinistre a une distribution gamma de paramètres $\alpha = 2$ et
  $\lambda = 0,5$ calculer l'espérance et la variance du montant total
  des sinistres, $S$.
  \begin{rep}
    $\esp{S} = 16$, $\var{S} = 160$
  \end{rep}
  \begin{sol}
    Nous avons $\esp{S} = \esp{N}\esp{X}$ et $\var{S} = \esp{N}\var{X} +
    \esp{X}^2\var{N}$. Or, $\esp{N} = 4 (0,5)/(1 - 0,5) = 4$, $\var{N}
    = 4 (0,5)/(1 - 0,5)^2 = 8$, $\esp{X} = 2/0,5 = 4$ et $\var{X} =
    2/0,5^2 = 8$, d'où
    \begin{align*}
        \esp{S}
        &= (4)(4)
         = 16 \\
        \intertext{et}
        \var{S}
        &= (4)(8) + (4^2)(8)
         = 160.
    \end{align*}
  \end{sol}
\end{exercice}

\begin{exercice}
  Interpréter l'inégalité suivante:
  \begin{displaymath}
    \Pr
    \left[
      0,96 \esp{S} \leq S \leq 1,04 \esp{S}
    \right]
    \geq 0,95.
  \end{displaymath}
  \begin{rep}
    L'inégalité signifie que l'on veut être certain à au moins $95$~\%
    que l'expérience individuelle d'un assuré ne varie pas de plus de
    $4$~\% autour de la prime pure.
  \end{rep}
\end{exercice}

\begin{exercice}
  \label{ex:stabilite:nPC}
  Dériver une formule générale pour déterminer le niveau de
  crédibilité complète d'ordre $(k, p)$ pour $\bar{S} = (S_1 + \dots +
  S_n)/n$ lorsque la distribution du montant total des sinistres est
  une Poisson composée.
  \begin{rep}
    $n \geq \lambda^{-1} (\zeta_{\varepsilon/2}/k)^2 (1 + \CV(X)^2)$
  \end{rep}
  \begin{sol}
    Puisque $\esp{\bar{S}} = \esp{S}$ et $\var{\bar{S}} = \var{S}/n$,
    l'inégalité
    \begin{displaymath}
      \Pr
      \left[
        (1 - k) \esp{\bar{S}} \leq \bar{S} \leq (1 + k) \esp{\bar{S}}
      \right]
      \geq p
    \end{displaymath}
    est satisfaite lorsque
    \begin{displaymath}
      n \ge
      \left( \frac{\zeta_{\varepsilon/2}}{k} \right)^2
      \frac{\var{S}}{\esp{S}^2}.
    \end{displaymath}
    Nous savons que dans le cas d'une Poisson composée,
    $\esp{S} = \lambda \esp{X}$ et
    $\var{S} = \lambda (\var{X} + \esp{X}^2)$. Ainsi,
    \begin{align*}
      n
      & \geq \left( \frac{\zeta_{\varepsilon/2}}{k} \right)^2
      \frac{\var{X} + \esp{X}^2}{\lambda \esp{X}^2} \\
      &= \frac{1}{\lambda} \left( \frac{\zeta_{\varepsilon/2}}{k}
      \right)^2 \left( 1 + \frac{\var{X}}{\esp{X}^2}
      \right) \\
      &= \frac{1}{\lambda} \left( \frac{\zeta_{\varepsilon/2}}{k}
      \right)^2 (1 + \CV(X)^2),
    \end{align*}
    où $\CV(X)$ est le coefficient de variation de $X$.
  \end{sol}
\end{exercice}

\begin{exercice}
  Soit $N \sim \text{Poisson}(256)$, $X \sim \text{Pareto}(3,\, 0,05)$
  et $S = X_1 + \dots + X_N$.
  \begin{enumerate}
  \item Quelle est la plus petite marge d'erreur admissible autour de
    $\esp{S}$ faisant toujours en sorte que $S$ a une crédibilité
    complète à 90~\%?  Interpréter brièvement le résultat.
  \item Quelle est la plus petite marge d'erreur admissible autour de
    $\esp{\bar{S}}$ faisant toujours en sorte que $\bar{S}$ a une
    crédibilité complète à 90~\% après 10 années?
  \end{enumerate}
  \begin{rep}
    \begin{inparaenum}
    \item $k \geq 0,2056$
    \item $k \geq 0,065$
    \end{inparaenum}
  \end{rep}
  \begin{sol}
    \begin{enumerate}
    \item Nous savons de l'\autoref{exemple:stabilite:comppois} que
      dans le cas d'une Poisson composée, la crédibilité complète est
      donnée par
      \begin{displaymath}
        \lambda \geq
        \left(
          \frac{\zeta_{\varepsilon/2}}{k}
        \right)^2
        \left(
          1 + \frac{\var{X}}{\esp{X}^2}
        \right).
      \end{displaymath}
      Or, nous avons $\esp{X} = \theta/(\alpha - 1) = 1/40$ et
      $\var{X} = \alpha \theta^2/((\alpha - 1)^2(\alpha - 2)) =
      3/\nombre{1600}$. Comme nous exigeons un niveau de crédibilité
      de $90$~\%, $\zeta_{\varepsilon/2} = 1,645$. En isolant $k$ dans
      la formule, nous obtenons
      \begin{displaymath}
        k \ge
        \frac{1,645}{\sqrt{256}}
        \sqrt{1 + \frac{3/\nombre{1600}}{1/\nombre{1600}}} =
        0,2056 = 20,56~\%.
      \end{displaymath}
      La valeur de $k$ est obtient une grande parce que la
      distribution du montant des sinistres est très asymétrique. Pour
      obtenir une valeur de $k$ plus usuelle (de l'ordre de $5$~\%),
      il faudrait augmenter le paramètre de Poisson, de manière à
      sommer un plus grand nombre de lois de Pareto.
    \item De manière similaire, mais en utilisant plutôt la formule
      \begin{displaymath}
        n \geq \frac{1}{\lambda}
        \left(
          \frac{\zeta_{\varepsilon/2}}{k}
        \right)^2
        \left(
          1 + \frac{\var{X}}{\esp{X}^2}
        \right)
      \end{displaymath}
      obtenue à l'\autoref{ex:stabilite:nPC}
      avec $n = 10$, nous obtenons
      \begin{displaymath}
        k \geq
        \frac{1}{\sqrt{256}} \frac{1,645}{\sqrt{10}}
        \sqrt{1 + \frac{3/\nombre{1600}}{1/\nombre{1600}}} = 0,065 = 6,5~\%.
      \end{displaymath}
      La marge d'erreur est plus petite qu'en a) puisque la
      distribution de $\bar{S}$ après 10 années est le résultat de la
      somme d'un beaucoup plus grand nombre de sinistres.
    \end{enumerate}
  \end{sol}
\end{exercice}

\begin{exercice}
  Sachant que la variable aléatoire $S$ de l'expérience des contrats
  obéit à une loi $N(\mu, \sigma^2)$, trouver, pour une période
  d'expérience, la relation entre $\mu$ et $\sigma$ pour avoir une
  crédibilité complète d'ordre $(k, p)$ pour chacune des combinaisons
  de $k$ et $p$ suivantes.
  \begin{enumerate}
  \item $(0,04, 0,95)$
  \item $(0,05, 0,90)$
  \item $(0,01, 0,98)$
  \end{enumerate}
  \begin{rep}
    \begin{inparaenum}
    \item $\mu \geq 49 \sigma$
    \item $\mu \geq 32,9 \sigma$
    \item $\mu \geq 232,6 \sigma$
    \end{inparaenum}
  \end{rep}
  \begin{sol}
    Dans un tel cas normale pure, la formule générale
    \eqref{eq:stabilite:complete} devient simplement
    \begin{displaymath}
      \mu \geq \left( \frac{\zeta_{\varepsilon/2}}{k} \right)
      \sigma.
    \end{displaymath}
    \begin{enumerate}
    \item On a $k = 0,04$ et $p = 0,95$ d'où $\zeta_{0,025} = 1,96$ et
      donc $\mu \geq 49 \sigma$.
    \item On a $k = 0,05$ et $p = 0,90$ d'où $\zeta_{0,05} = 1,645$ et
      donc $\mu \geq 32,9 \sigma$.
    \item On a $k = 0,01$ et $p = 0,98$ d'où $\zeta_{0,01} = 2,326$ et
      donc $\mu \geq 232,6 \sigma$.
    \end{enumerate}
  \end{sol}
\end{exercice}

\begin{exercice}
  Sachant que $S \sim N(\mu, \sigma^2)$ et $\bar{S} = (S_1 + \dots +
  S_9)/9$, où $S_1, \dots, S_9$ sont toutes des variables aléatoires
  mutuellement indépendantes distribuées comme $S$, déterminer la
  relation entre $\mu$ et $\sigma$ faisant en sorte que $\bar{S}$ a
  une crédibilité complète d'ordre $(k, p)$ pour chacune des
  combinaisons de $k$ et $p$ suivantes.
  \begin{enumerate}
  \item (0,05, 0,90)
  \item (0,05, 0,95)
  \item (0,01, 0,90)
  \item (0,01, 0,95)
  \end{enumerate}
  \begin{rep}
    \begin{inparaenum}
    \item $\mu \geq 10,97 \sigma$
    \item $\mu \geq 13,07 \sigma$
    \item $\mu \geq 54,83 \sigma$
    \item $\mu \geq 65,33 \sigma$
    \end{inparaenum}
  \end{rep}
  \begin{sol}
    On nous donne que $\bar{S} \sim N(\mu, \sigma^2/9)$. La
    crédibilité complète d'ordre $(k, p)$ est donc atteinte lorsque
    \begin{displaymath}
      \esp{\bar{S}} \geq \left( \frac{\zeta_{\varepsilon/2}}{k} \right)
      \sqrt{\var{\bar{S}}},
    \end{displaymath}
    soit
    \begin{displaymath}
      \mu \geq \left( \frac{\zeta_{\varepsilon/2}}{k} \right)
      \frac{\sigma}{3}.
    \end{displaymath}
    \begin{enumerate}
    \item On a $k = 0,05$ et $p = 0,90$ d'où $\zeta_{0,05} = 1,645$ et
      $\mu \geq 10,97 \sigma$.
    \item On a $k = 0,05$ et $p = 0,95$ d'où $\zeta_{0,025} = 1,96$ et
      $\mu \geq 13,07 \sigma$.
    \item On a $k = 0,01$ et $p = 0,90$ d'où $\zeta_{0,05} = 1,645$ et
      $\mu \geq 54,83 \sigma$.
    \item On a $k = 0,01$ et $p = 0,95$ d'où $\zeta_{0,025} = 1,96$ et
      $\mu \geq 65,33 \sigma$.
    \end{enumerate}
  \end{sol}
\end{exercice}

\begin{exercice}
  \label{ex:stabilite:rBNC}
  Juliette travaille au département de tarification d'une compagnie
  d'assurance très active en assurance automobile. À la suite d'une
  analyse de données exhaustive, Juliette peut affirmer que la
  fréquence des sinistres pour ce type de produit a une distribution
  binomiale négative de paramètres $r$ et $\theta = 0,01$. La sévérité
  des sinistres, quant à elle, suit une loi gamma de paramètres
  $\alpha = 0,02$ et $\lambda = 1$. Trouver la plus petite valeur de
  $r$ telle que le montant total des sinistres d'un contrat
  d'assurance automobile sera à plus ou moins $5$~\% égal à sa moyenne,
  $19$ fois sur $20$.
  \begin{rep}
    $r \geq \nombre{2328,24}$
  \end{rep}
  \begin{sol}
    Déterminons tout d'abord une formule générale pour le cas de la
    binomiale négative composée. Sachant que
    $\esp{N} = r(1 - \theta)/\theta$ et que
    $\var{N} = r (1 - \theta)/\theta^2$, nous obtenons
    \begin{displaymath}
      r \geq
      \left(
        \frac{\zeta_{\varepsilon/2}}{k}
      \right)^2
      \left(
        \frac{1}{1 - \theta} +
        \frac{\theta}{1-\theta} \frac{\var{X}}{\esp{X}^2}
      \right).
    \end{displaymath}
    Or, dans cet exercice, $\theta = 0,01$, $\var{X}/\esp{X}^2 =
    1/\alpha = 50$ et $\zeta_{\varepsilon/2}/k = 1,96/0,05 = 39,2$. Par
    conséquent, $r \geq \nombre{2328,24}$.
  \end{sol}
\end{exercice}

\begin{exercice}
  Soit $S_j$ le montant total des sinistres pour un assuré à la
  période $j = 1, \dots, n$ tel que
  \begin{displaymath}
    S_j = X_1 + X_2 + \dots + X_{N_j},
  \end{displaymath}
  où $X_1, X_2, \dots$ sont les montants individuels des sinistres
  dont la distribution est dégénérée en $M$ et $N_j$ suit une loi de
  Poisson de paramètre $\lambda$. Calculer le nombre total de
  sinistres espéré minimal pour accorder une crédibilité complète
  d'ordre $(0,04, 0,90)$ à l'expérience individuelle $\bar{S}$.
  \begin{rep}
    \nombre{1692}
  \end{rep}
  \begin{sol}
    Remarquez tout d'abord que le contexte est celui de
    l'\autoref{ex:stabilite:nPC}, sauf que le critère de crédibilité
    complète est exprimé non pas en nombre d'années d'expérience, $n$,
    mais plutôt en nombre total de sinistres espéré, soit $n \lambda$.
    Nous avons donc
    \begin{displaymath}
      n \lambda \geq
      \left(
        \frac{\zeta_{\varepsilon/2}}{k}
      \right)^2
      (1 + \CV(X)^2).
    \end{displaymath}
    Or, dans le cas présent, $\CV(X) = 0$ puisque $\Pr[X = M]
    = 1$ (distribution dégénérée en $M$), $k = 0,04$ et $\zeta_{0,05}
    = 1,645$, d'où $n \lambda \geq \nombre{1692}$.
  \end{sol}
\end{exercice}

\begin{exercice}
  Soit $S_{j}$ le montant total des sinistres pour un assuré à la
  période $j = 1, \dots, n$ et
  $S_{j} \sim \text{Poisson composée}(\lambda, F_X(\cdot))$. Le
  montant d'un sinistre individuel a une variance de $100$ et une
  espérance de $5$. Déterminer le nombre minimal espéré de sinistres
  pour accorder une crédibilité complète selon les critères
  ci-dessous.
  \begin{enumerate}
  \item Le montant total des sinistres demeure à $3$~\% ou moins du
    montant espéré avec une probabilité de $95$~\%.
  \item Le nombre total de sinistres demeure à $3$~\% ou moins du nombre
    espéré avec une probabilité de $95$~\%.
  \end{enumerate}
  \begin{rep}
    \begin{inparaenum}
    \item $\nombre{21343}$ sinistres
    \item $\nombre{4269}$ sinistres
    \end{inparaenum}
  \end{rep}
  \begin{sol}
    \begin{enumerate}
    \item Le montant total des sinistres a une distribution Poisson
      composée avec $\esp{X} = 5$, $\var{X} = 100$, $k = 0,03$ et
      $\zeta_{\varepsilon/2} = \zeta_{0,025} = 1,96$. De la formule
      générale \eqref{eq:stabilite:complete_comppois}, nous obtenons
      directement $\lambda \geq \nombre{21343}$.
    \item La distribution du nombre de sinistres est une Poisson. Nous
      pouvons utiliser la formule
      \eqref{eq:stabilite:complete_comppois} avec $\var{X} = 0$, ce
      qui donne $\lambda \geq \nombre{4269}$.
    \end{enumerate}
  \end{sol}
\end{exercice}

\begin{exercice}
  Une compagnie assure deux groupes ayant la même loi de Poisson pour
  la fréquence de leurs sinistres individuels. Cependant, les
  individus du groupe A ne peuvent avoir que des sinistres de $50$
  alors que les individus du groupe B ont des sinistres obéissant à
  une loi gamma de moyenne $50$. Sachant que l'observation de
  \nombre{1000} sinistres est suffisante pour accorder une crédibilité
  complète au groupe A et que \nombre{3000} sinistres sont nécessaires
  pour accorder une crédibilité complète de même ordre au groupe B,
  calculer les paramètres de la loi gamma en question.
  \begin{rep}
    $\alpha = 1/2$, $\lambda = 1/100$
  \end{rep}
  \begin{sol}
    Soit $X_A$ la variable aléatoire du montant des sinistres du
    groupe A et $X_B$ la variable aléatoire du montant des sinistres
    du groupe B. On nous donne $\esp{X_A} = 50$, $\var{X_A} = 0$ et
    $X_B \sim \text{Gamma}(\alpha, \lambda)$ avec
    $\alpha/\lambda = 50$. De plus, la distribution du montant total
    des sinistres est une Poisson composée, donc le seuil de
    crédibilité complète est donné par \eqref{eq:stabilite:complete_comppois}.

    Pour le groupe A $\lambda_A = \nombre{1000}$ et $\CV(X_A) = 0$,
    d'où $(\zeta_{\varepsilon/2}/k)^2 = \nombre{1000}$. Ce facteur est
    identique pour le groupe B, mais $\lambda_B = \nombre{3000}$ et
    $\CV(X_B) = 1/\sqrt{\alpha}$, d'où
    \begin{displaymath}
      \nombre{3000} = \nombre{1000}
      \left(
        1 + \frac{1}{\alpha}
      \right),
    \end{displaymath}
    soit $\alpha = 1/2$. Enfin, de $\alpha/\lambda = 50$ découle que
    $\lambda = 1/100$.
  \end{sol}
\end{exercice}

\begin{exercice}
  Dans un modèle Poisson composée, on accorde une crédibilité complète
  d'ordre $(0,05, p)$ au nombre total de sinistres si le nombre de
  sinistres observé est supérieur à $\nombre{1000}$. Quel doit être le
  nombre minimal de sinistres pour accorder une crédibilité d'ordre
  $(0,25, p)$ au montant total des sinistres si le montant d'un
  sinistre individuel suit une loi Gamma$(2, 2)$?
  \begin{rep}
    60 sinistres
  \end{rep}
  \begin{sol}
    Deux situations sont décrites:
    \begin{enumerate}[1)]
    \item la crédibilité complète est accordée à la fréquence des
      sinistres seulement;
    \item la crédibilité complète est accordée au montant total des
      sinistres.
    \end{enumerate}
    On nous donne donc d'abord le seuil de crédibilité complète dans
    un modèle avec distribution de Poisson pure (ou Poisson composée
    avec une distribution des montants de sinistres dégénérée en 1),
    puis on nous demande de déterminer le seuil dans un modèle avec
    distribution Poisson composée où la distribution du montant des
    sinistres a un coefficient de variation de $1/\sqrt{2}$. Toujours
    à partir de la formule \eqref{eq:stabilite:complete_comppois},
    nous trouvons que, dans un premier temps,
    \begin{displaymath}
      \nombre{1000} =
      \left(
        \frac{\zeta_{\varepsilon/2}}{0,05}
      \right)^2,
    \end{displaymath}
    soit $\zeta_{\varepsilon/2} = 0,05 \sqrt{\nombre{1000}}$. Dans un
    deuxième temps, nous avons
    \begin{align*}
      \lambda
      &\geq
      \left(
        \frac{\zeta_{\varepsilon/2}}{0,25}
      \right)^2
      \left(
        1 + \frac{1}{2}
      \right) \\
      &= 1,5
      \left(
        \frac{0,05 \sqrt{\nombre{1000}}}{0,25}
      \right)^2 \\
      &= 60.
    \end{align*}
  \end{sol}
\end{exercice}

\begin{exercice}
  Pour ses calculs relatifs à l'admission au régime rétrospectif, la
  CNESST peut n'employer que la fréquence des sinistres, ou encore la
  fréquence ainsi que la sévérité. Dans le premier cas, le nombre
  minimal d'employés pour être admissible au régime rétrospectif est
  de $\nombre{6494}$ en supposant que la probabilité d'avoir un accident
  est de $0,04$.  Sachant que le coefficient de variation de la sévérité
  des sinistres est $2$, que devient le seuil d'admissibilité
  lorsque la sévérité est également prise en compte dans les calculs?
  \begin{rep}
    $\nombre{33552}$
  \end{rep}
  \begin{sol}
    Il faut d'abord identifier le modèle pour la fréquence comme étant
    une $\text{Binomiale}(n, \theta)$, où $n$ est le nombre d'employés
    et $\theta$, la probabilité qu'un employé ait un accident. Nous
    devons étudier deux cas: la binomiale pure et la binomiale
    composée. La formule de crédibilité de stabilité générale est
    \begin{displaymath}
      \esp{S} \geq
      \left(
        \frac{\zeta_{\varepsilon/2}}{k}
      \right) \sqrt{\var{S}}.
    \end{displaymath}
    Si $S \sim \text{Binomiale}(n, \theta)$, alors
    \begin{displaymath}
      n \geq
      \left(
        \frac{\zeta_{\varepsilon/2}}{k}
      \right)^2 \frac{1 - \theta}{\theta}
    \end{displaymath}
    et, si $S \sim \text{Binomiale composée}(n, \theta, F_X(\cdot))$,
    \begin{displaymath}
      n \geq
      \left(
        \frac{\zeta_{\varepsilon/2}}{k}
      \right)^2
      \left(
        \frac{1}{\theta}\, \CV(X)^2 +
        \frac{1 - \theta}{\theta}
      \right).
    \end{displaymath}
    (À noter que cette dernière équation revient à la première si
    $X = 1$ avec probabilité $1$, c'est-à-dire si $\CV(X) = 0$.) Nous
    avons $\theta = 0,04$ et, dans le cas d'une binomiale pure, un
    niveau de crédibilité complète de $n = \nombre{6494}$. Par
    conséquent, $(\zeta_{\varepsilon/2}/k)^2 = 270,5833$. En insérant
    cette valeur dans la formule pour la binomiale composée avec
    $\CV(X) = 2$, nous obtenons un niveau de crédibilité complète de
    $n = \nombre{33552}$ employés.
  \end{sol}
\end{exercice}

\begin{exercice}
  Le montant total des sinistres a une distribution Poisson composée
  où les montants de sinistres individuels proviennent d'une loi de
  Pareto de paramètres $\alpha = 3$ et $\lambda = 100$. Lorsque la
  largeur de l'intervalle de confiance autour de la moyenne est $5$~\%
  de celle-ci, le seuil de crédibilité complète est $\nombre{2500}$. On
  décide de changer la distribution de la fréquence des sinistres pour
  une binomiale négative avec $\theta = 0,5$. Si le seuil de
  crédibilité complète et le niveau de confiance de l'intervalle
  autour de la moyenne demeurent tous deux inchangés, quelle est la
  nouvelle largeur de l'intervalle de confiance?
  \begin{rep}
    $\nombre{13975}$
  \end{rep}
  \begin{sol}
    Avec la distribution Poisson composée, le seuil de crédibilité
    complète est
    \begin{displaymath}
      \lambda \geq
      \left( \frac{\zeta_{\varepsilon/2}}{k} \right)^2
      \left( 1 + \frac{\var{X}}{\esp{X}^2} \right).
    \end{displaymath}
    Or, on nous donne $\esp{X} = 100/(3 - 1) = 50$ et $\var{X} = 3
    \cdot 100^2/((3 - 1)^2 (3 - 2)) = \nombre{7500}$. Si le seuil de
    crédibilité complète dans le cas Poisson composée est
    $\nombre{2500}$ quand $k = 0,05$, alors
    \begin{displaymath}
      \zeta_{\varepsilon/2}
      = 0,05 \sqrt{\frac{2500}{1 + 7500/50^2}}
      = 1,25.
    \end{displaymath}
    Dans le cas binomiale négative composée, le seuil de crédibilité
    complète est plutôt donné par
    \begin{displaymath}
      r \geq
      \left(
        \frac{\zeta_{\varepsilon/2}}{k}
      \right)^2
      \left(
        \frac{1}{1 - \theta} +
        \frac{\theta}{1-\theta} \frac{\var{X}}{\esp{X}^2}
      \right).
    \end{displaymath}
    Avec $\var{X}/\esp{X}^2 = 3$, $\theta = 0,5$, $r = \nombre{2500}$
    et $\zeta_{\varepsilon/2} = 1,25$, nous obtenons $k = 0,0559$, d'où la
    largeur de l'intervalle de confiance autour de la moyenne est $2 k
    \esp{S} = 2 (0,559) (\nombre{2500}) (50) = \nombre{13975}$.
  \end{sol}
\end{exercice}

\begin{exercice}
  On vous donne les renseignements suivants:
  \begin{enumerate}[(i)]
  \item $S = \sum_{j=1}^{N} X_j$ et les variables aléatoires
    $X_j$ sont mutuellement indépendantes et indépendantes de $N$.
  \item $X_j \sim \text{Pareto}(3, 3)$
  \item $N \sim \text{Binomiale négative}(r, 1/3)$.
  \end{enumerate}
  Calculer la valeur minimale de $r$ pour accorder une crédibilité
  complète d'ordre $(0,05,\, 0,90)$ à $S$. Utiliser l'approximation
  normale.
  \begin{rep}
    $r \geq 3247,23$
  \end{rep}
  \begin{sol}
    Nous avons un cas binomiale négative composée. À l'aide de la
    formule développée à l'\autoref{ex:stabilite:rBNC}, nous obtenons
    directement que
    \begin{align*}
      r
      &\geq
      \left(
        \frac{\zeta_{\varepsilon/2}}{k}
      \right)^2
      \left(
        \frac{1}{1 - \theta} +
        \frac{\theta}{1-\theta} \frac{\var{X}}{\esp{X}^2}
      \right) \\
      &=
      \left(
        \frac{1,645}{0,05}
      \right)^2
      \left(
        \frac{3}{2} + \frac{1}{2} \cdot 3
      \right) \\
      &= \nombre{3247,23}.
    \end{align*}
  \end{sol}
\end{exercice}

%%%
%%% CRÉDIBILITÉ PARTIELLE
%%%

\begin{exercice}
  Soit $S$ la variable aléatoire du montant total des sinistres pour
  un portefeuille d'assurance. Sachant que $S \sim \text{Poisson
    composée}(\lambda, F_X(\cdot))$, trouver le seuil de crédibilité
  complète en termes du nombre espéré de sinistres selon les
  hypothèses de distribution suivantes.
  \begin{enumerate}
  \item $\Pr[X = 1] = 1$ (dégénérée en $1$), $k = 0,05$ et $p = 0,90$.
  \item $X \sim \text{Exponentielle}(2)$, $k = 0,04$, $p = 0,95$.
  \item Si le seuil de crédibilité complète selon les conditions en b)
    n'est pas atteint, quel facteur de crédibilité partielle
    accorderait-on à $\bar{S}$ pour une période d'expérience (en
    fonction de $\lambda$)?
  \end{enumerate}
  \begin{rep}
    \begin{inparaenum}
    \item $\nombre{1082}$
    \item $\nombre{4802}$
    \item $\sqrt{\lambda/\nombre{4802}}$
    \end{inparaenum}
  \end{rep}
  \begin{sol}
    \begin{enumerate}
    \item Avec $X \equiv 1$, $S$ est une Poisson pure et le seuil de
      crédibilité complète est $\lambda \geq
      (\zeta_{\varepsilon/2}/k)^2 = (1,645/0,05)^2 = 1082,41$.
    \item Dans le cas Poisson composée avec $\CV(X) = 1$, le
      seuil de crédibilité complète est $\lambda \geq
      (\zeta_{\varepsilon/2}/k)^2 (1 + \CV(X)^2) = 2
      (1,96/0,04)^2 = \nombre{4802}$.
    \item Sans plus de précisions sur la formule de crédibilité
      partielle à utiliser, nous allons utiliser sur la formule de la
      racine carrée. Pour une seule période d'expérience, nous avons
      $\bar{S} = S$, donc la crédibilité partielle $z$ serait
      \begin{displaymath}
        z = \min
        \left\{
          \sqrt{\frac{\lambda}{\nombre{4802}}}, 1
        \right\}.
      \end{displaymath}
    \end{enumerate}
  \end{sol}
\end{exercice}

\begin{exercice}
  L'actuaire de la compagnie ABC croit qu'il faudrait \nombre{3000}
  sinistres pour accorder une crédibilité complète à un assuré d'un
  groupe si la sévérité est constante.  Après une étude, il se rend
  compte que la sévérité suit plutôt une loi de Pareto de moyenne
  $\nombre{1000}$ avec $\alpha = 3$. Si le nombre de sinistres obéit à
  une loi de Poisson, combien de sinistres doit-on avoir observé si
  l'on a accordé, après une période d'expérience, une crédibilité de
  $0,5$ à l'expérience individuelle de l'assuré?
  \begin{rep}
    \nombre{3000} sinistres
  \end{rep}
  \begin{sol}
    Nous avons un modèle Poisson composée pour le montant total des
    sinistres. Le seuil de crédibilité complète de \nombre{3000} est
    atteint lorsque $\CV(X) = 0$, donc
    $(\zeta_{\varepsilon/2}/k)^2 = \nombre{3000}$. Si nous avons plutôt
    $\CV(X)^2 = \alpha/(\alpha - 2) = 3$, alors le seuil de
    crédibilité complète est $\lambda_0 = \nombre{3000}(1 + 3) =
    \nombre{12000}$. Enfin, si l'on a accordé à l'assuré une
    crédibilité partielle de $z = \sqrt{\lambda/\nombre{12000}} = 0,5$,
    alors c'est que l'assuré a eu $\lambda = \nombre{3000}$
    sinistres.
  \end{sol}
\end{exercice}

\begin{exercice}
  On vous donne les informations suivantes:
  \begin{enumerate}[(i)]
  \item Le nombre de sinistres suit une loi de Poisson.
  \item Le montant des sinistres a une distribution log-normale avec un
    coefficient de variation de 3.
  \item Le nombre et les montants de sinistres sont indépendants.
  \item Le nombre de sinistres la première année fut de \nombre{1000}.
  \item Le montant total des sinistres la première année fut de
    $6,75$~millions.
  \item La prime collective pour la seconde année est de
    $5,00$~millions.
  \item Le volume du contrat est le même pour la première et la
    seconde année.
  \item Le niveau de crédibilité complète assure que le montant total
    des sinistres sera à $5$~\% de la moyenne $95$~\% du temps.
  \end{enumerate}
  Déterminer la prime de crédibilité (partielle) selon l'approche de
  crédibilité de stabilité. Calculer le niveau de crédibilité complète
  sur le nombre d'années d'expérience et utiliser la formule de la
  racine carrée pour calculer le facteur de crédibilité.
  \begin{rep}
    $5,45$
  \end{rep}
  \begin{sol}
    Il faut déduire de l'énoncé que la crédibilité complète est donnée
    en années d'expérience et que $\zeta_{\varepsilon/2} = 1,96$, $k =
    0,05$, $\CV(X) = 3$ et $\lambda = \nombre{1000}$. Puisque
    nous avons un modèle Poisson composée, le niveau de crédibilité
    complète $n_0$ est
    \begin{displaymath}
      n_0 =
      \frac{1}{\nombre{1000}}
      \left( \frac{1,96}{0,05} \right)^2
      (1 + (3)^2) =
      15,3664.
    \end{displaymath}
    Après $n = 1$ année d'expérience, le facteur de crédibilité est
    \begin{equation*}
      z = \sqrt{1/15,3664} = 0,2551.
    \end{equation*}
    Puisque $\bar{S} = S_1 = 6,75$ et $m = 5,00$, la prime de
    crédibilité partielle est
    \begin{displaymath}
      P = 0,2551 (6,75) + (1 - 0,2551) (5) = 5,45.
    \end{displaymath}
  \end{sol}
\end{exercice}

\begin{exercice}
  On vous dit que $S \sim \text{Poisson composée}(\lambda,
  F_X(\cdot))$. La fréquence annuelle espérée des sinistres est
  évaluée à $0,035$. La grandeur minimale du portefeuille de
  l'assureur pour accorder une crédibilité complète à l'expérience est
  $\nombre{103500}$. Pour accorder une crédibilité de $0,67$ à
  l'expérience d'une période, quelle doit être la grandeur minimale du
  portefeuille?
  \begin{rep}
    $\nombre{46462}$
  \end{rep}
  \begin{sol}
    Soit $\lambda$ la grandeur du portefeuille. Le seuil de
    crédibilité complète est $\lambda_0 = \nombre{103500}$. En
    utilisant la formule de la racine carrée, on a donc
    \begin{displaymath}
      z = 0,67 = \sqrt{\frac{\lambda}{\nombre{103500}}},
    \end{displaymath}
    d'où $\lambda = \nombre{46462}$.
  \end{sol}
\end{exercice}

\begin{exercice}
  Les contrats d'une compagnie d'assurance pour un certain type de
  produit ont les caractéristiques ci-dessous.
  \begin{center}
    \begin{tabular}{lcc}
      \toprule
      & Nombre de sinistres & Montant des sinistres \\
      \midrule
      Espérance & 10 & \nombre{5000} \\
      Variance  & 10 & \nombre{6250000} \\
      \bottomrule
    \end{tabular}
  \end{center}
  Une crédibilité complète est accordée à l'expérience d'un contrat
  après $n$ années si celle-ci se concentre dans un intervalle de
  $10$~\% autour de sa moyenne avec probabilité de $90$~\%. Déterminer
  après combien d'années d'expérience le facteur de crédibilité
  partielle sera de $0,54$ selon chacune des formules de crédibilité
  partielle ci-dessous.
  \begin{enumerate}
  \item $z = \left( \dfrac{n}{n_0} \right)^{1/2}$
  \item $z = \left( \dfrac{n}{n_0} \right)^{2/3}$
  \end{enumerate}
  \begin{rep}
    \begin{inparaenum}
    \item $9,86$ années
    \item $13,42$ années
    \end{inparaenum}
  \end{rep}
  \begin{sol}
    Nous avons $\esp{S} = \esp{N} \esp{X} = 50,000$ et
    $\var{S} = \var{N} \esp{X}^2 + \var{X} \esp{N} =
    \nombre{312500000}$. Avec la formule générale
    \eqref{eq:stabilite:complete_comppois}, le seuil de crédibilité
    complète est
    \begin{align*}
      n_0
      &= \left( \frac{\zeta_{\varepsilon/2}}{k} \right)^2
      \frac{\var{S}}{\esp{S}} \\
      &= 270,6025 (0,125) \\
      &= 33,8253.
    \end{align*}
    \begin{enumerate}
    \item Avec la formule de la racine carrée,
      \begin{displaymath}
        0,54 = \sqrt{\frac{n}{33,8253}} \Rightarrow n = 9,86.
      \end{displaymath}
    \item Avec la formule «puissance 2/3»,
      \begin{displaymath}
        0,54 = \left( \frac{n}{33,8253} \right)^{2/3} \Rightarrow n = 13,42.
      \end{displaymath}
      Le temps requis pour atteindre la pleine crédibilité est donc
      plus élevé avec cette formule.
    \end{enumerate}
  \end{sol}
\end{exercice}

\Closesolutionfile{solutions}
\Closesolutionfile{reponses}

%%% Local Variables:
%%% mode: latex
%%% TeX-engine: xetex
%%% TeX-master: "theorie-credibilite-avec-r"
%%% coding: utf-8
%%% End:

\include{bayesienne}
\include{buhlmann}
\include{buhlmann-straub}

\appendix
\chapter{Formules de crédibilité exacte}
\label{chap:formules}

Le tableau de la page suivante contient les principaux résultats de
crédibilité bayesienne pour les combinaisons de distributions de la
famille exponentielle univariée. Le contenu des colonnes du tableau
est le suivant, dans l'ordre:
\begin{quote}
  \begin{enumerate}
  \item La distribution de $S|\Theta = \theta$;
  \item La distribution a priori de $\Theta$;
  \item La distribution marginale de $S$;
  \item La distribution a posteriori de $\Theta|S_1 = x_1, \dots, S_n
    = x_n$, toujours du même type que la distribution a priori, mais
    avec des paramètres mis à jour;
  \item La prime de risque $\mu(\theta) = \esp{S|\Theta = \theta}$;
  \item La prime collective $m = \esp{\mu(\Theta)}$;
  \item La prime bayesienne $B_{n+1} = \esp{\mu(\Theta)|S_1, \dots,
      S_n}$, toujours égale à la prime collective évaluée avec les
    paramètres de la distribution a posteriori;
  \item Le facteur de crédibilité dans l'expression de la prime
    bayesienne sous forme de prime de crédibilité.
  \end{enumerate}
\end{quote}

Les fonctions de masse ou de densité de probabilité des lois figurant
au tableau se trouvent à l'annexe \ref{chap:distributions}.

\newpage

\begin{center}
  \begin{sideways}
    \begin{tabular}{lllp{30mm}cccc}
      \toprule
      %% Entêtes
      \multicolumn{1}{c}{$f(x|\theta)$} &
      \multicolumn{1}{c}{$u(\theta)$} &
      \multicolumn{1}{c}{$f(x)$} &
      \multicolumn{1}{c}{$u(\theta|x_1, \dots, x_n)$} &
      $\mu(\theta)$ &
      $m$ &
      $B_{n+1}$ &
      $z$ \\
      \midrule
      %% Cas Bernoulli/bêta
      Bernoulli$(\theta)$ &
      Bêta$(\alpha, \beta)$ &
      Bernoulli$(\alpha/(\alpha + \beta))$ &
      Bêta$(\tilde{\alpha}, \tilde{\beta})$ \newline
      $\tilde{\alpha} = \alpha + \sum_t x_t$ \newline
      $\tilde{\beta} = \beta + n - \sum_t x_t$ &
      $\theta$ &
      $\dfrac{\alpha}{\alpha + \beta}$ &
      $\dfrac{\alpha + \sum_t S_t}{\alpha + \beta + n}$ &
      $\dfrac{n}{n + \alpha + \beta}$ \\
      \midrule
      %% Cas géometrique/bêta
      Géometrique$(\theta)$ &
      Bêta$(\alpha, \beta)$ &
      $\dfrac{\Gamma(\alpha + \beta) \Gamma(\alpha + 1) \Gamma(\beta + x)}
      {\Gamma(\alpha) \Gamma(\beta) \Gamma(\alpha + \beta + x + 1)}$ &
      Bêta$(\tilde{\alpha}, \tilde{\beta})$ \newline
      $\tilde{\alpha} = \alpha + n$ \newline
      $\tilde{\beta} = \beta + \sum_t x_t$ &
      $\dfrac{1 - \theta}{\theta}$ &
      $\dfrac{\beta}{\alpha - 1}$ &
      $\dfrac{\beta + \sum_t S_t}{\alpha + n - 1}$ &
      $\dfrac{n}{n + \alpha - 1}$ \\
      \midrule
      %% Cas Poisson/gamma
      Poisson$(\theta)$ &
      Gamma$(\alpha, \lambda)$ &
      Bin.\ nég.$(\alpha, \lambda/(\lambda + 1))$ &
      Gamma($\tilde{\alpha}$, $\tilde{\lambda}$) \newline
      $\tilde{\alpha} = \alpha + \sum_t x_t$ \newline
      $\tilde{\lambda} = \lambda + n$ &
      $\theta$ &
      $\dfrac{\alpha}{\lambda}$ &
      $\dfrac{\alpha + \sum_t S_t}{\lambda + n}$ &
      $\dfrac{n}{n + \lambda}$ \\
      \midrule
      %% Cas exponentielle/gamma
      Exponentielle$(\theta)$ &
      Gamma$(\alpha, \lambda)$ &
      Pareto$(\alpha, \lambda)$ &
      Gamma$(\tilde{\alpha}, \tilde{\lambda})$ \newline
      $\tilde{\alpha} = \alpha + n$ \newline
      $\tilde{\lambda} = \lambda + \sum_t x_t$ &
      $\dfrac{1}{\theta}$ &
      $\dfrac{\lambda}{\alpha - 1}$ &
      $\dfrac{\lambda + \sum_t S_t}{\alpha + n - 1}$ &
      $\dfrac{n}{n + \alpha - 1}$ \\
      \midrule
      %% Cas normale/normale
      Normale$(\theta, \sigma_2^2)$ &
      Normale$(\mu, \sigma_1^2)$ &
      Normale$(\mu, \sigma_1^2 + \sigma_2^2)$ &
      Normale$(\tilde{\mu}, \tilde{\sigma}_1^2)$ \newline
      $\tilde{\mu} =
      \dfrac{\sigma_1^2 \sum_t x_t + \sigma_2^2 \mu}{n \sigma_1^2 +
        \sigma_2^2}$ \newline
      $\tilde{\sigma}_1^2 = \dfrac{\sigma_1^2 \sigma_2^2}{n
        \sigma_1^2 + \sigma_2^2}$ &
      $\theta$ &
      $\mu$ &
      $\dfrac{\sigma_1^2 \sum_t S_t + \sigma_2^2 \mu}{n \sigma_1^2 +
        \sigma_2^2}$ &
      $\dfrac{n}{n + \sigma_2^2/\sigma_1^2}$ \\
      \bottomrule
    \end{tabular}
  \end{sideways}
\end{center}

%%% Local Variables:
%%% mode: latex
%%% TeX-master: "exercices_credibilite"
%%% coding: utf-8
%%% End:

\chapter{Paramétrisation des lois de probabilité}
\label{chap:distributions}

%% Liste à puce compacte et sans puce, justement. Pour cette annexe
%% seulement.
\begingroup
\setlist[itemize]{label={},leftmargin=0pt,align=left,nosep}

Cette annexe précise la paramétrisation des lois de probabilité
utilisée dans les énoncés des exercices des chapitres
\ref{chap:stabilite} à \ref{chap:buhlmann-straub} ainsi que dans le
tableau des formules de crédibilité exacte de l'annexe
\ref{chap:formules}; la racine des noms de fonctions pour calculer les
caractéristiques des lois dans R; l'espérance, la variance et la
fonction génératrice des moments (lorsqu'elle existe) des différentes
lois.

La prise en charge de certaines lois dans R requiert le paquetage
\pkg{actuar} \citep{actuar}.


\section{Distributions discrètes}

\subsection{Binomiale}
\label{distributions:binomiale}

\begin{itemize}
\item Racine: \code{binom}
\item Paramètres: \code{size} ($n = 0, 1, 2, \dots$),
  \code{prob} ($0 \leq \theta \leq 1$)
\end{itemize}
\begin{itemize}
\item Cas spécial: Bernoulli($\theta)$ lorsque $n = 1$.
\end{itemize}
\begin{align*}
  \Pr[X = x]
  &= \binom{n}{x} \theta^x (1 - \theta)^{n - x}, \quad
  x = 0, 1, \dots \\
  \esp{X}
  &= n \theta \\
  \var{X}
  &= n \theta (1 - \theta) \\
  M(t)
  &= (1 - \theta + \theta e^t)^n
\end{align*}

\subsection{Binomiale négative}
\label{distributions:binomialeneg}

\begin{itemize}
\item Racine: \code{nbinom}
\item Paramètres: \code{size} ($r \geq 0$),
  \code{prob} ($0 < \theta \leq 1$),
  \code{mu} ($r(1 - \theta)/\theta$)
\end{itemize}
\begin{itemize}
\item Cas spécial: Géométrique($\theta)$ lorsque $r = 1$.
\end{itemize}
\begin{align*}
  \Pr[X = x]
  &= \binom{x + r - 1}{r - 1} \theta^r (1 - \theta)^x, \quad
  x = 0, 1, \dots \\
  \esp{X}
  &= \frac{r (1 - \theta)}{\theta} \\
  \var{X}
  &= \frac{r (1 - \theta)}{\theta^2} \\
  M(t)
  &= \left( \frac{\theta}{1 - (1 - \theta)e^t} \right)^r
\end{align*}

\subsection{Poisson}
\label{distributions:poisson}

\begin{itemize}
\item Racine: \code{pois}
\item Paramètre: \code{lambda} ($\lambda \geq 0$)
\end{itemize}
\begin{align*}
  \Pr[X = x]
  &= \frac{\lambda^x e^{-\lambda}}{x!}, \quad x = 0, 1, \dots \\
  \esp{X}
  &= \lambda \\
  \var{X}
  &= \lambda \\
  M(t)
  &= e^{\lambda (e^t - 1)}
\end{align*}


\section{Distributions continues}

\subsection{Bêta}
\label{distributions:beta}

\begin{itemize}
\item Racine: \code{beta}
\item Paramètres: \code{shape1} ($a > 0$),
      \code{shape2} ($b > 0$)
\end{itemize}
\begin{itemize}
\item Cas spécial: Uniforme$(0, 1)$ lorsque $\alpha = \beta = 1$.
\end{itemize}
\begin{align*}
  f(x)
  &= \frac{\Gamma(a + b)}{\Gamma(a) \Gamma(b)}\,
  x^{a - 1} (1 - x)^{b - 1}, \quad 0 < x < 1 \\
  \esp{X}
  &= \frac{a}{a + b} \\
  \var{X}
  &= \frac{a b}{(a + b)^2 (a + b + 1)}
\end{align*}


\subsection{Gamma}
\label{distributions:gamma}

\begin{itemize}
\item Racine: \code{gamma}
\item Paramètres: \code{shape} ($\alpha > 0$),
      \code{rate}   ($\lambda = 1/\theta$),
      \code{scale}  ($\theta > 0$)
\end{itemize}
\begin{itemize}
\item Cas spéciaux: Exponentielle$(\lambda)$ lorsque $\alpha = 1$,
  $\chi^2(r)$ lorsque $\alpha = r/2$ et $\lambda = 1/2$.
\end{itemize}
\begin{align*}
  f(x)
  &= \frac{\lambda^\alpha}{\Gamma(\alpha)}\, x^{\alpha - 1} e^{-\lambda x},
  \quad x > 0 \\
  \esp{X}
  &= \frac{\alpha}{\lambda} \\
  \var{X}
  &= \frac{\alpha}{\lambda^2} \\
  M(t)
  &= \left( \frac{\lambda}{\lambda - t} \right)^\alpha
\end{align*}


\subsection{Normale}
\label{distributions:normale}
\begin{itemize}
\item Racine: \code{norm}
\item Paramètres: \code{mean} ($-\infty < \mu < \infty$), \code{sd}
  ($\sigma > 0$)
\end{itemize}
\begin{align*}
  f(x)
  &= \frac{1}{\sqrt{2\pi} \sigma}\,
  \exp\left\{ -\frac{(x - \mu)^2}{2\sigma^2} \right\}, \quad
  -\infty < x < \infty \\
  \esp{X}
  &= \mu \\
  \var{X}
  &= \sigma^2 \\
  M(t)
  &= e^{\mu t + \sigma^2 t^2/2}
\end{align*}

\subsection{Log-normale}
\label{distributions:log-normale}

Si $X \sim \text{Normale}(\mu, \sigma^2)$, alors $Y = e^X \sim
\text{Log-normale}(\mu, \sigma^2)$.
\begin{itemize}
\item Racine: \code{lnorm}
\item Paramètres: \code{meanlog} ($-\infty < \mu < \infty$), \code{sdlog}
  ($\sigma > 0$)
\end{itemize}
\begin{align*}
  f(x)
  &= \frac{1}{\sqrt{2\pi} \sigma} \frac{1}{x}\,
  \exp\left\{ -\frac{(\ln x - \mu)^2}{2\sigma^2} \right\}, \quad
  x > 0 \\
  \esp{X}
  &= e^{\mu + \sigma^2/2} \\
  \var{X}
  &= e^{2\mu + \sigma^2}(e^{\sigma^2} - 1) \\
\end{align*}

\subsection{Pareto}
\label{distributions:pareto}

\begin{itemize}
\item Racine: \code{pareto}, \code{pareto2} [\pkg{actuar}]
\item Paramètres: \code{shape} ($\alpha > 0$),
      \code{scale}  ($\lambda > 0$)
\end{itemize}
\begin{align*}
  f(x) &=
  \frac{\alpha \lambda^\alpha}{(x + \lambda)^{\alpha+1}},
  \quad x > 0 \\
  \esp{X}
  &= \frac{\lambda}{\alpha - 1}, \quad \alpha > 1 \\
  \var{X}
  &= \frac{\alpha \lambda^2}{(\alpha - 1)^2 (\alpha - 2)}, \quad
  \alpha > 2
\end{align*}

\subsection{Pareto généralisée}
\label{distributions:paretogen}

\begin{itemize}
\item Racine: \code{genpareto} [\pkg{actuar}]
\item Paramètres: \code{shape1} ($\alpha > 0$),
      \code{shape2} ($\tau > 0$),
      \code{rate}   ($\lambda = 1/\theta$),
      \code{scale}  ($\theta > 0$)
\end{itemize}
Paramètres: $\alpha > 0$, $\lambda > 0$, $\tau > 0$. Cas spécial:
Pareto$(\alpha, \lambda)$ lorsque $\tau = 1$.
\begin{align*}
  f(x) &=
  \frac{\Gamma(\alpha + \tau)}{\Gamma(\alpha) \Gamma(\tau)}\,
  \frac{\theta^\alpha x^{\tau - 1}}{(x + \theta)^{\alpha+\tau}},
  \quad x > 0 \\
  \esp{X}
  &= \frac{\theta \tau}{\alpha - 1}, \quad \alpha > 1 \\
  \var{X}
  &= \frac{\theta^2 \tau (\alpha + \tau - 1)}{(\alpha - 1)^2 (\alpha
    - 2)}, \quad \alpha > 2
\end{align*}


\section{Distributions composées}

Soit $S = X_1 + \dots + X_N$, où $X_1, \dots, X_n$ sont des variables
aléatoires mutuellement indépendantes, identiquement distribuées et
toutes indépendantes de $N$. On a toujours
\begin{align*}
  \esp{S} &= \esp{N} \esp{X} \\
  \var{S} &= \var{N} \esp{X}^2 + \esp{N} \var{X}.
\end{align*}

\subsection{Binomiale composée}
\label{distributions:binomialecomposée}

Distribution de $S$ lorsque $N \sim \text{Binomiale}(n, \theta)$ et
$\Pr[X \leq x] = F_X(x)$.
\begin{align*}
  \esp{S} &= n \theta \esp{X} \\
  \var{S} &= n\theta(1 - \theta) \esp{X}^2 + n \theta \var{X}
\end{align*}

\subsection{Binomiale négative composée}
\label{distributions:binomialenegativecomposée}

Distribution de $S$ lorsque $N \sim \text{Binomiale négative}(r,
\theta)$ et $\Pr[X \leq x] = F_X(x)$.
\begin{align*}
  \esp{S} &= \frac{r(1 - \theta)}{\theta}\, \esp{X} \\
  \var{S} &= \frac{r(1 - \theta)}{\theta^2}\, \esp{X}^2 + \frac{r(1 -
    \theta)}{\theta}\, \var{X}
\end{align*}

\subsection{Poisson composée}
\label{distributions:poissoncomposée}

Distribution de $S$ lorsque $N \sim \text{Poisson}(\lambda)$ et $\Pr[X
\leq x] = F_X(x)$.
\begin{align*}
  \esp{S} &= \lambda \esp{X} \\
  \var{S} &= \lambda \esp{X^2}
\end{align*}

\endgroup

%%% Local Variables:
%%% mode: latex
%%% TeX-engine: xetex
%%% TeX-master: "theorie-credibilite-avec-r"
%%% coding: utf-8
%%% End:

\chapter{Solutions}
\label{chap:solutions}

\input{solutions-stabilite}
\input{solutions-bayesienne}
\input{solutions-buhlmann}
\input{solutions-buhlmann-straub}

%%% Local Variables:
%%% mode: latex
%%% TeX-engine: xetex
%%% TeX-master: "theorie-credibilite-avec-r"
%%% coding: utf-8
%%% End:


\bibliography{cred,actu,stat,r,vg}

\cleardoublepage
\printindex

\pagestyle{empty}

\cleartoverso
\input{colophon}

\cleartoverso
\input{couverture-arriere}

\end{document}

%%% Local Variables:
%%% mode: latex
%%% TeX-engine: xetex
%%% TeX-master: t
%%% coding: utf-8
%%% End:
